%%%%%%%%%%%%%%%%%%%%%%%%%%%%%%%%%%%%%%%%%%%%%%%%%%%%%%%%%%%%%%%%%%%%%%%%%%%%%%%%%%%%%%%%%%%%%%%%%%%%%%%%%%%%%%%%%%%%%%%%%%%%%%%%%%%%%%%%%%%%%%%%%%%%%%%%%%%%%%%%%%%
% Written By Michael Brodskiy
% Class: Electricity & Magnetism
% Professor: D. Wood
%%%%%%%%%%%%%%%%%%%%%%%%%%%%%%%%%%%%%%%%%%%%%%%%%%%%%%%%%%%%%%%%%%%%%%%%%%%%%%%%%%%%%%%%%%%%%%%%%%%%%%%%%%%%%%%%%%%%%%%%%%%%%%%%%%%%%%%%%%%%%%%%%%%%%%%%%%%%%%%%%%%

\include{Includes.tex}

\title{Homework C}
\date{December 6, 2023}
\author{Michael Brodskiy\\ \small Professor: D. Wood}

\begin{document}

\maketitle

\newpage

\begin{abstract}

  The purpose of this document is to analyze how the electric field of a parallel plate capacitor may be plotted in the GNU Octave program. Various scenarios will be analyzed, such as a basic conductor (with air), a conductor with a dielectric, and a conductor with a diamagnetic.

\end{abstract}

\begin{flushleft}
  Keywords: \underline{parallel plate capacitor}, \underline{electric field}, \underline{GNU Octave}, \underline{dielectric}, \underline{diamagnetic}
\end{flushleft}

\end{document}

