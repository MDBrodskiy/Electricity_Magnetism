%%%%%%%%%%%%%%%%%%%%%%%%%%%%%%%%%%%%%%%%%%%%%%%%%%%%%%%%%%%%%%%%%%%%%%%%%%%%%%%%%%%%%%%%%%%%%%%%%%%%%%%%%%%%%%%%%%%%%%%%%%%%%%%%%%%%%%%%%%%%%%%%%%%%%%%%%%%%%%%%%%%
% Written By Michael Brodskiy
% Class: Electricity & Magnetism
% Professor: D. Wood
%%%%%%%%%%%%%%%%%%%%%%%%%%%%%%%%%%%%%%%%%%%%%%%%%%%%%%%%%%%%%%%%%%%%%%%%%%%%%%%%%%%%%%%%%%%%%%%%%%%%%%%%%%%%%%%%%%%%%%%%%%%%%%%%%%%%%%%%%%%%%%%%%%%%%%%%%%%%%%%%%%%

\documentclass[12pt]{article} 
\usepackage{alphalph}
\usepackage[utf8]{inputenc}
\usepackage[russian,english]{babel}
\usepackage{titling}
\usepackage{amsmath}
\usepackage{graphicx}
\usepackage{enumitem}
\usepackage{amssymb}
\usepackage[super]{nth}
\usepackage{everysel}
\usepackage{ragged2e}
\usepackage{geometry}
\usepackage{multicol}
\usepackage{fancyhdr}
\usepackage{cancel}
\usepackage{siunitx}
\usepackage{physics}
\usepackage{tikz}
\usepackage{mathdots}
\usepackage{yhmath}
\usepackage{cancel}
\usepackage{color}
\usepackage{array}
\usepackage{multirow}
\usepackage{gensymb}
\usepackage{tabularx}
\usepackage{extarrows}
\usepackage{booktabs}
\usepackage{lastpage}
\usetikzlibrary{fadings}
\usetikzlibrary{patterns}
\usetikzlibrary{shadows.blur}
\usetikzlibrary{shapes}

\geometry{top=1.0in,bottom=1.0in,left=1.0in,right=1.0in}
\newcommand{\subtitle}[1]{%
  \posttitle{%
    \par\end{center}
    \begin{center}\large#1\end{center}
    \vskip0.5em}%

}
\usepackage{hyperref}
\hypersetup{
colorlinks=true,
linkcolor=blue,
filecolor=magenta,      
urlcolor=blue,
citecolor=blue,
}


\title{Homework 5}
\date{October 19, 2023}
\author{Michael Brodskiy\\ \small Professor: D. Wood}

\begin{document}

\maketitle

\begin{enumerate}

  \item Four point charges are located a distance $a$ from the origin on the $y$ and $z$ axes as shown.  Find the approximate expression for the electric \textbf{potential} far from the charges.  Use spherical coordinates and retain the only the \textbf{first} non-vanishing terms in the multipole expansion. [Hint: consider breaking down the distribution into a superposition of individual dipoles.]

    We can write the dipole moment from the configuration of charges:

    $$p=-2q(-a_y)+3q(a_z)-2q(a_y)+q(-a_z)$$
    $$=\cancel{2qaa_y}+3qaa_z\cancel{-2qa(a_y)}-aqa_z$$
    $$=2qa_z$$

    We know that:

    $$V_{mon}(r)=\frac{Q}{4\pi\varepsilon_o r}=0$$
    $$V_{dip}(r)=\frac{\vec{p}\bold{\hat{r}}}{4\pi\varepsilon_o r^2}$$

    Thus, we may write the dipole as:

    $$V_{dip}(r)=\frac{2qa_z\bold{\hat{r}}}{4\pi\varepsilon_o r^2}$$

    This can finally be written in spherical coordinates using:

    $$\boxed{V_{dip}(r,\theta)\approx\frac{qa\cos(\theta)}{2\pi\varepsilon_o r^2}}$$

  \item Three point charges are located a distance $a$ from the origin on the $y$ and $z$ axes as shown.  Find the approximate expression for the electric \textbf{field} far from the charges.  Use spherical coordinates and retain the \textbf{first two} non-vanishing terms in the multipole expansion.

    We begin by working from the lowest order up, which is the monopole. First we find the aggregate charge:

    $$Q=-q+q-q=-q$$

    This can simply be plugged into the formula:

    $$V_{mon}(r)=-\frac{q}{4\pi\varepsilon_o r}$$

    We then find the dipole contribution, beginning with the moment:

    $$p=-q(-a_y)+q(a_z)-q(a_y)$$
    $$=\cancel{qa_y}+qa_z\cancel{-qa_y}$$
    $$=qa_z$$

    We then use the formula:

    $$V_{dip}(r)=\frac{\vec{p}\bold{\hat{r}}}{4\pi\varepsilon_o r^2}$$
    $$V_{dip}(r)=\frac{qa_z\bold{\hat{r}}}{4\pi\varepsilon_o r^2}$$
    $$V_{dip}(r,\theta)=\frac{qa\cos(\theta)}{4\pi\varepsilon_o r^2}$$

    Summing the two contributions, we get:

    $$V(r,\theta)\approx-\frac{q}{4\pi\varepsilon_o r}+\frac{qa\cos(\theta)}{4\pi\varepsilon_or^2}$$

    We now use the formula:

    $$\vec{E}=-\vec{\nabla}V$$
    $$\vec{E}=\vec{\nabla}\left(  \frac{q}{4\pi\varepsilon_or}-\frac{qa\cos(\theta)}{4\pi\varepsilon_or^2}\right)$$
    $$\vec{E}=\frac{\partial}{\partial r}\left(  \frac{q}{4\pi\varepsilon_or}-\frac{qa\cos(\theta)}{4\pi\varepsilon_or^2}\right)+\frac{1}{r}\frac{\partial}{\partial\theta}\left(-\frac{qa\cos(\theta)}{4\pi\varepsilon_or^2}\right)$$
    $$\vec{E}=\left(  -\frac{q}{4\pi\varepsilon_or^2}+\frac{qa\cos(\theta)}{2\pi\varepsilon_or^3}\right)\bold{\hat{r}}+\left(\frac{qa\sin(\theta)}{4\pi\varepsilon_or^3}\right)\bold{\hat{\theta}}$$

    Finally, this yields:

    $$\boxed{\vec{E}(r,\theta)=\frac{q}{2\pi\varepsilon_or^2}\left[ \left( -\frac{1}{2}+\frac{a\cos(\theta)}{r} \right)\bold{\hat{r}} +\left( \frac{a\sin(\theta)}{2r} \right)\bold{\hat{\theta}}\right]}$$

  \item For the charged spherical shell in Problem 4 of Assignment 4 (the one with $V(R,\theta)=V_o\sin^2(\theta)$ and $\sigma=\frac{V_o\varepsilon_o}{3R}(7-15\cos^2(\theta)$), find the monopole and dipole moments.

      First, we compute the monopole moment. This is done by first finding the total charge:

      $$q_{tot}=\int\sigma(R,\theta)\,da$$
      $$da'=2\pi R^2\sin(\theta)\,d\theta$$

      We can then compute the integral:

      $$q=\int_0^\pi\frac{V_o\varepsilon_o}{3R}[7-15\cos^2(\theta)](2\pi R^2\sin(\theta))\,d\theta$$
      $$q=\frac{2V_o\varepsilon_o\pi R}{3}\int_0^\pi 7\sin(\theta)-15\sin(\theta)\cos^2(\theta)\,d\theta$$
      $$q=\frac{2V_o\varepsilon_o\pi R}{3}\int_0^\pi -8\sin(\theta)+15\sin^3(\theta)\,d\theta$$

      Using a numerical solver, we obtain:

      $$q=\frac{2V_o\varepsilon_o\pi R}{3}\left( -7\cos(\theta)+5\cos^3(\theta)\right)\Big|_0^\pi$$
      $$q=\frac{4V_o\varepsilon_o\pi R}{3}-\left( -\frac{4V_o\varepsilon_o\pi R}{3} \right)$$

      And finally, we find:

      $$\boxed{q_{tot}=\frac{8V_o\varepsilon_o\pi R}{3}}$$

      Now, we find the dipole moment. The integral set-up becomes very similar, except that:

      $$\vec{p}=p\bold{\hat{z}}=\int\vec{r}\bold{\hat{z}}\sigma\,da$$

      This can be converted to:

      $$p\bold{\hat{z}}=\int_0^\pi\frac{V_o\varepsilon_o}{3R}[7-15\cos^2(\theta)](2\pi R^2\sin(\theta))(R\cos(\theta))\,d\theta$$
      $$=\frac{2V_o\varepsilon_o R^2\pi}{3}\int_0^\pi[7-15\cos^2(\theta)](\sin(\theta))(\cos(\theta))\,d\theta$$
      $$=\frac{V_o\varepsilon_o R^2\pi}{3}\int_0^\pi 7\sin(2\theta)-15\sin(2\theta)\cos^2(\theta)\,d\theta$$

      Again implementing a numerical solver, we obtain:

      $$=\frac{V_o\varepsilon_o R^2\pi}{3}\left( -\frac{7}{2}\cos(2\theta)+\frac{15}{2}\cos^4(\theta)\right)\Big|_0^\pi$$
      $$=\frac{V_o\varepsilon_o R^2\pi}{3}\left( -\frac{7}{2}+\frac{15}{2}-\left[ -\frac{7}{2}+\frac{15}{2} \right]\right)$$
      $$=0$$

      Thus, we see:

      $$\boxed{\left\{\begin{array}{l}\vec{p}_{mon}=\dfrac{8V_o\varepsilon_o\pi R}{3}\\\vec{p}_{dip}=0\end{array}}$$

  \item A thin rod on the $z$-axis goes from $z=-a$ to $z=+a$ and carries a linear charge density of $\lambda(z)$. Find the leading term in the multipole expansion for:

    \begin{enumerate}

      \item $\lambda(z)=\lambda_o\cos\left( \frac{\pi z}{a} \right)$

        We can write the formula for the multipole expansion as:

        $$V=\frac{1}{4\pi\varepsilon_o}\sum_{n=0}^\infty \frac{P_n(\cos(\theta))}{r^{n+1}}\int_{-a}^az^n\lambda(z)\,dz$$

        For the $n=0$ case, we can write:

        $$V=\frac{1}{4\pi\varepsilon_o r}\int_{-a}^a \lambda(z)\,dz$$
        $$V=\frac{\lambda_0}{4\pi\varepsilon_o r}\int_{-a}^a \cos\left( \frac{\pi z}{a} \right)\,dz$$

        From this, we can see that the integral expression would evaluate to zero, meaning we have to try the next term. At $n=1$, we get:

        $$V=\frac{\cos(\theta)}{4\pi\varepsilon_or^2}\int_{-a}^a z\lambda(z)\,dz$$
        $$V=\frac{\lambda_o\cos(\theta)}{4\pi\varepsilon_or^2}\int_{-a}^a z\cos\left( \frac{\pi z}{a} \right)\,dz$$
        $$V=\frac{\lambda_o\cos(\theta)}{4\pi\varepsilon_or^2}\left(\underbrace{ \cancel{\frac{a}{\pi}z\sin\left( \frac{\pi z}{a} \right)}}_{0}+\frac{a^2}{\pi^2}\cos\left( \frac{\pi z}{a} \right) \right)\Big|_{-a}^a$$
        $$V=\frac{\lambda_o\cos(\theta)}{4\pi\varepsilon_or^2}\left( \frac{a^2}{\pi^2}\cos\left( \frac{\pi z}{a} \right) \right)\Big|_{-a}^a$$

        Once again, we see that the integral expression evaluates to zero. Thus, we move up another order to $n=2$:

        $$V_{quad}(r,\theta)=\frac{1}{4\pi\varepsilon_o}\frac{3\cos^2(\theta)-1}{2r^3}\int_{-a}^a z^2\lambda(z)\,dz$$
        $$V_{quad}(r,\theta)=\frac{\lambda_o}{4\pi\varepsilon_o}\frac{3\cos^2(\theta)-1}{2r^3}\int_{-a}^a z^2\cos\left( \frac{\pi z}{a} \right)\,dz$$

        Using a numerical solver, we calculate the integral:

        $$V_{quad}(r,\theta)=\frac{\lambda_o}{4\pi\varepsilon_o}\frac{3\cos^2(\theta)-1}{2r^3}\underbrace{\left(\frac{z^2a}{\pi}\sin\left( \frac{\pi z}{a} \right)+\frac{2a^2z}{\pi^2}\cos\left( \frac{\pi z}{a} \right)-\frac{2a^3}{\pi^2}\sin\left( \frac{\pi z}{a} \right) \right)\right\Big|_{-a}^a}_{-\dfrac{4a^3}{\pi^2}}$$

        Thus, the leading term becomes:

        $$\boxed{V_{quad}(r,\theta)=-\frac{a^3\lambda_o\left( 3\cos^2(\theta)-1 \right)}{2\pi^3\varepsilon_or^3}}$$

      \item $\lambda(z)=\lambda_1\cos\left( \frac{\pi z}{2a} \right)$

        where $\lambda_o$ and $\lambda_1$ are constants.

        For $\lambda(z)=\lambda_1\cos\left( \frac{\pi z}{2a} \right)$, we can begin by finding the first term, according to the formula from (a), at $n=0$:

        $$V_{mon}=\frac{1}{4\pi\varepsilon_o}\frac{1}{r}\int_{-a}^a\lambda(z)\,dz$$
        $$V_{mon}=\frac{1}{4\pi\varepsilon_o}\frac{\lambda_1}{r}\int_{-a}^a\cos\left( \frac{\pi z}{2a} \right)\,dz$$

        We can then evaluate the integral:

        $$\int_{-a}^a\cos\left( \frac{\pi z}{2a} \right)\,dz=\frac{2a}{\pi}\sin\left( \frac{\pi z}{2a} \right)\Big|_{-a}^{a}$$
        $$=\frac{4a}{\pi}$$

        Which then becomes:

        $$\frac{\lambda_1}{4\pi\varepsilon_o r}\left( \frac{4a}{\pi} \right)$$

        And finally:

        $$\boxed{V_{mon}(r)=\frac{a\lambda_1}{\pi^2\varepsilon_or}}$$

    \end{enumerate}

\end{enumerate}

\end{document}

