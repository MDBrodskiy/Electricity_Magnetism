%%%%%%%%%%%%%%%%%%%%%%%%%%%%%%%%%%%%%%%%%%%%%%%%%%%%%%%%%%%%%%%%%%%%%%%%%%%%%%%%%%%%%%%%%%%%%%%%%%%%%%%%%%%%%%%%%%%%%%%%%%%%%%%%%%%%%%%%%%%%%%%%%%%%%%%%%%%%%%%%%%%
% Written By Michael Brodskiy
% Class: Electricity & Magnetism
% Professor: D. Wood
%%%%%%%%%%%%%%%%%%%%%%%%%%%%%%%%%%%%%%%%%%%%%%%%%%%%%%%%%%%%%%%%%%%%%%%%%%%%%%%%%%%%%%%%%%%%%%%%%%%%%%%%%%%%%%%%%%%%%%%%%%%%%%%%%%%%%%%%%%%%%%%%%%%%%%%%%%%%%%%%%%%

\documentclass[12pt]{article} 
\usepackage{alphalph}
\usepackage[utf8]{inputenc}
\usepackage[russian,english]{babel}
\usepackage{titling}
\usepackage{amsmath}
\usepackage{graphicx}
\usepackage{enumitem}
\usepackage{amssymb}
\usepackage[super]{nth}
\usepackage{everysel}
\usepackage{ragged2e}
\usepackage{geometry}
\usepackage{multicol}
\usepackage{fancyhdr}
\usepackage{cancel}
\usepackage{siunitx}
\usepackage{physics}
\usepackage{tikz}
\usepackage{mathdots}
\usepackage{yhmath}
\usepackage{cancel}
\usepackage{color}
\usepackage{array}
\usepackage{multirow}
\usepackage{gensymb}
\usepackage{tabularx}
\usepackage{extarrows}
\usepackage{booktabs}
\usepackage{lastpage}
\usetikzlibrary{fadings}
\usetikzlibrary{patterns}
\usetikzlibrary{shadows.blur}
\usetikzlibrary{shapes}

\geometry{top=1.0in,bottom=1.0in,left=1.0in,right=1.0in}
\newcommand{\subtitle}[1]{%
  \posttitle{%
    \par\end{center}
    \begin{center}\large#1\end{center}
    \vskip0.5em}%

}
\usepackage{hyperref}
\hypersetup{
colorlinks=true,
linkcolor=blue,
filecolor=magenta,      
urlcolor=blue,
citecolor=blue,
}


\title{Homework 4}
\date{October 12, 2023}
\author{Michael Brodskiy\\ \small Professor: D. Wood}

\begin{document}

\maketitle

\begin{enumerate}

  \item Consider an infinite grounded conducting plane bent at a $90^{\circ}$ angle between the $yz$ and $xz$ planes as shown, with a charge placed at $x = 4a$, $y = a$. Use appropriate image charge(s) to find an expression for the potential $V(x,y,z)$ in the region $x > 0$, $y > 0$.

  \item The boundary at $x = 0$ consists of two metal strips: one, from $y = 0$ to $y = a/2$ is held at a constant potential $+V_0$ and the other, from $y = 2/a$ to $y = a$ is held at a constant potential of $−V_0$ . Solve for the potential $V(x,y,z)$ inside the slot. Feel free to use the relevant results from Example 3.3 or from lecture as a starting point.

  \item Consider a long (semi-infinite) rectangular conducting pipe oriented $V_0$ parallel to the $z$-axis, with dimensions $a\times b$ in the $xy$-plane. The pipe itself is grounded, and the rectangle at the closed end is at a constant potential $V_0$ . Find an expression for the potential everywhere inside the pipe (for $z > 0$).

  \item Consider an empty spherical shell of charge of radius $R$ where the potential on the surface is given by $V(R, \theta) = V_o\sin^2(\theta)$.

    Hint: Express $\sin^2(\theta)$ as a polynomial function of $\cos(\theta)$.

    \begin{enumerate}

      \item Find $V(r, \theta)$ inside the shell.

      \item Find $\vec{E}(R,\theta)$ just inside the shell.

      \item Find $V(r, \theta)$ out of the shell.

      \item Find $\vec{E}(R, \theta)$ just outside the shell.

      \item Find $\sigma(R, \theta)$ on the shell. [answer: $\sigma = \frac{V_o\varepsilon_o}{3R}(7 - 15\cos^2(\theta))$]

    \end{enumerate}

  \item An empty spherical shell of radius $R$ has potential $V_0$ on the upper hemisphere and $−V_0$ on the lower hemisphere

    \begin{enumerate}

      \item Calculate the first two non-zero terms of the expression for the potential outside of the sphere to obtain an approximate expression for $V(r, \theta)$ in this region.

      \item From this approximate expression, compute the value of $V(R, \theta)$ (on the surface of the shell) for $\theta = 0, \theta = \pi/4,$ and $\theta = 3\pi/4$ compare the results with the exact values at those locations

    \end{enumerate}

\end{enumerate}

\end{document}

