%%%%%%%%%%%%%%%%%%%%%%%%%%%%%%%%%%%%%%%%%%%%%%%%%%%%%%%%%%%%%%%%%%%%%%%%%%%%%%%%%%%%%%%%%%%%%%%%%%%%%%%%%%%%%%%%%%%%%%%%%%%%%%%%%%%%%%%%%%%%%%%%%%%%%%%%%%%%%%%%%%%
% Written By Michael Brodskiy
% Class: Electricity & Magnetism
% Professor: D. Wood
%%%%%%%%%%%%%%%%%%%%%%%%%%%%%%%%%%%%%%%%%%%%%%%%%%%%%%%%%%%%%%%%%%%%%%%%%%%%%%%%%%%%%%%%%%%%%%%%%%%%%%%%%%%%%%%%%%%%%%%%%%%%%%%%%%%%%%%%%%%%%%%%%%%%%%%%%%%%%%%%%%%

\include{Includes.tex}

\title{Homework 1}
\date{\today}
\author{Michael Brodskiy\\ \small Professor: D. Wood}

\begin{document}

\maketitle

\begin{enumerate}
    
  \item Calculate:

    \begin{enumerate}

      \item $\vec{\nabla}\left( \frac{1}{r} \right)$

        We know $r=\sqrt{x^2+y^2+z^2}$. Converting and applying the chain rule, we get:

        $$\frac{\partial}{\partial x}\left( x^2+y^2+z^2\right)^{-\frac{1}{2}}\bold{\hat{x}}+\frac{\partial}{\partial y}\left( x^2+y^2+z^2\right)^{-\frac{1}{2}}\bold{\hat{y}}+\frac{\partial}{\partial z}\left( x^2+y^2+z^2\right)^{-\frac{1}{2}}\bold{\hat{z}}$$
        $$(x^2+y^2+z^2)^{-\frac{3}{2}}(x\bold{\hat{x}}+y\bold{\hat{y}}=z\bold{\hat{z}})$$
        $$\frac{x\bold{\hat{x}}+y\bold{\hat{y}}+z\bold{\hat{z}}}{r^3}$$

        We also know that $\vec{r}=x\bold{\hat{x}}+y\bold{\hat{y}}+z\bold{\hat{z}}$, and that $r\bold{\hat{r}}=\vec{r}$. Thus, we get:

        $$\boxed{\vec{\nabla}\left( \frac{1}{r} \right)=\frac{\vec{r}}{r^3}\rightarrow\frac{\bold{\hat{r}}}{r^2}}$$

      \item $\vec{\nabla}\cdot\bold{\hat{x}}$

        This implies the following:

        $$v_x=1\bold{\hat{x}},\quad v_y=0\bold{\hat{y}},\quad v_z=0\bold{\hat{z}}$$

        Thus, we find:

        $$\boxed{\vec{\nabla}\cdot\bold{\hat{x}}=\frac{\partial}{\partial x}(1\bold{\hat{x}})=0}$$

      \item $\vec{\nabla}\cdot\bold{\hat{r}}$

        This could be written as:

        $$\vec{\nabla}\cdot\left( \frac{x\bold{\hat{x}}+y\bold{\hat{y}}+z\bold{\hat{z}}}{\sqrt{x^2+y^2+z^2}} \right)$$

        Thus, we need to compute:

        $$\frac{\partial}{\partial x}\left(\frac{x}{\sqrt{x^2+y^2+z^2}}\right)$$

        with respect to each variable. By the quotient rule, we find

        $$\frac{\partial}{\partial x}\left(\frac{x}{\sqrt{x^2+y^2+z^2}}\right)\rightarrow \frac{\sqrt{x^2+y^2+z^2}-x^2(x^2+y^2+z^2)^{-0.5}}{x^2+y^2+z^2}$$

        Converting back to $r$, we obtain:

        $$\frac{1}{r}-\frac{x^2}{r^3}$$

        By symmetry, we know that the corresponding $y$ and $z$ variables become:

        $$\frac{1}{r}-\frac{y^2}{r^3}\quad\text{and}\quad\frac{1}{r}-\frac{z^2}{y^3}$$

        Summing the results, we get:

        $$\left( \frac{1}{r}-\frac{x^2}{r^3} \right)+\left( \frac{1}{r}-\frac{y^2}{r^3} \right)+\left( \frac{1}{r}-\frac{z^2}{r^3} \right)$$
        $$\frac{3}{r}-\left( \frac{x^2+y^2+z^2}{r^3} \right)$$
        $$\boxed{\vec{\nabla}\cdot\bold{\hat{r}}=\frac{3}{r}-\frac{1}{r}=\frac{2}{r}}$$

      \item $\vec{\nabla}r^n$ ($n>0$)

        $$\vec{\nabla}((x^2+y^2+z^2)^\frac{n}{2})\Rightarrow \frac{n}{2}(x^2+y^2+z^2)^{\frac{n}{2}-1}\left( 2x\bold{\hat{x}}+2y\bold{\hat{y}}+2z\bold{\hat{z}} \right)$$
        $$n\vec{r}r^{n-2}\rightarrow \frac{n\vec{r}}{r^{2-n}}$$

        Thus, we get:

        $$\boxed{\frac{n\bold{\hat{r}}}{r^{1-n}}\quad\text{or}\quad n\bold{\hat{r}}r^{n-1}}$$

    \end{enumerate}

  \item Calculate the divergence and curls of the following functions:

    \begin{enumerate}

      \item $\vec{v}_a=xy\bold{\hat{x}}+yz\bold{\hat{y}}+zy\bold{\hat{z}}$

        \begin{itemize}

          \item Divergence:

            $$\vec{\nabla}\cdot \vec{v}_a=\frac{\partial}{\partial x}(xy)+\frac{\partial}{\partial y}(yz)+\frac{\partial}{\partial z}(zy)$$
            $$\boxed{\text{div}(\vec{v}_a)=y+z+y=2y+z}$$

          \item Curl:

            $$\vec{\nabla}\times\vec{v}_a=\left|\begin{matrix} \bold{\hat{x}} & \bold{\hat{y}} & \bold{\hat{z}}\\ \frac{\partial}{\partial x} & \frac{\partial}{\partial y} & \frac{\partial}{\partial z} \\ xy & yz & zy \end{matrix}\right|=(z-y)\bold{\hat{x}}-(0-0)\bold{\hat{y}}+(0-x)\bold{\hat{z}}$$
            $$\boxed{\text{curl}(\vec{v}_a)=\langle z-y, 0, -x\rangle}$$

        \end{itemize}

      \item $\vec{v}_b=y^2\bold{\hat{x}}+(2xy+z^2)\bold{\hat{y}}+2xz\bold{\hat{z}}$

        \begin{itemize}

          \item Divergence:

            $$\vec{\nabla}\cdot\vec{v}_b=\frac{\partial}{\partial x}(y^2)+\frac{\partial}{\partial y}(2xy+z^2)+\frac{\partial}{\partial z}(2xz)$$
            $$\boxed{\text{div}(\vec{v}_b)=0+2x+2x=4x}$$

          \item Curl:

            $$\vec{\nabla}\times\vec{v}_b=\left|\begin{matrix} \bold{\hat{x}} & \bold{\hat{y}} & \bold{\hat{z}}\\ \frac{\partial}{\partial x} & \frac{\partial}{\partial y} & \frac{\partial}{\partial z} \\ y^2 & 2xy+z^2 & 2xz \end{matrix}\right|=(0-2z)\bold{\hat{x}}-(2z-0)\bold{\hat{y}}+(2y-2y)\bold{\hat{z}}$$
            $$\boxed{\text{curl}(\vec{v}_b)=\langle -2z, -2z, 0\rangle}$$

        \end{itemize}

      \item $\vec{v}_c=yz\bold{\hat{x}}+xz\bold{\hat{y}}+xy\bold{\hat{z}}$

        \begin{itemize}

          \item Divergence:

            $$\vec{\nabla}\cdot\vec{v}_c=\frac{\partial}{\partial x}(yz)+\frac{\partial}{\partial y}(xz)+\frac{\partial}{\partial z}(xy)$$
            $$\boxed{\text{div}(\vec{v}_c)=0+0+0=0}$$

          \item Curl:

            $$\vec{\nabla}\times\vec{v}_c=\left|\begin{matrix} \bold{\hat{x}} & \bold{\hat{y}} & \bold{\hat{z}}\\ \frac{\partial}{\partial x} & \frac{\partial}{\partial y} & \frac{\partial}{\partial z} \\ yz & xz & xy \end{matrix}\right|=(x-x)\bold{\hat{x}}-(y-y)\bold{\hat{y}}+(z-z)\bold{\hat{z}}$$
            $$\boxed{\text{curl}(\vec{v}_c)=\langle 0, 0, 0\rangle}$$

        \end{itemize}

    \end{enumerate}

  \item Calculate the following. Note that some results should be vectors.  It is helpful to check the dimensionality of your answers:

    \begin{enumerate}

      \item $(\vec{r}\cdot\vec{\nabla})\vec{r}$

        $$(\vec{r}\cdot\vec{\nabla})=\frac{\partial}{\partial x}(x\bold{\hat{x}})+\frac{\partial}{\partial y} (y\bold{\hat{y}})+\frac{\partial}{\partial z}(z\bold{\hat{z}})=1+1+1=3$$
        $$\boxed{(\vec{r}\cdot\vec{\nabla})\vec{r}=3\vec{r}=3x\bold{\hat{x}}+3y\bold{\hat{y}}+3z\bold{\hat{z}}}$$

      \item $(\bold{\hat{r}}\cdot\vec{\nabla})r$

        From Problem \textbf{1c}, we know the value of $(\bold{\hat{r}}\cdot\vec{\nabla})$, which gives us:

        $$\boxed{\left( \frac{2}{r} \right)r=2}$$

      \item $(\bold{\hat{r}}\cdot\vec{\nabla})\bold{\hat{r}}$

        Again, employing what we know from Problem \textbf{1c}, we get:

        $$\boxed{\left( \frac{2}{r} \right)\bold{\hat{r}}=\frac{2\bold{\hat{r}}}{r}}$$

    \end{enumerate}

  \item Consider a vector field $\vec{v}=2xz\bold{\hat{x}}+(x+2)\bold{\hat{y}}+y(z^2-3)\bold{\hat{z}}$ and a cube with vertices at (0,0,0), (0,2,0), (2,0,0), (0,0,2), etc. In the text (Example 1.7), the flux was calculated through the top five faces of the cube, and found to total to 20.

    \begin{enumerate}

      \item Find the upward flux through the bottom surface, i.e.\ the square surface in the $xy$-plane bounded by the points (0,0,0), (2,0,0), (2,2,0), (0,2,0).

        According to Green's Theorem, we know:

        $$\iiint \vec{\nabla}\cdot\vec{N}\,d\tau=\iint\vec{v}\cdot d\vec{a}$$

        Furthermore, because the surface is in the $xy$ plane, and the flux we want to find is in the $\bold{\hat{z}}$ direction, we know:

        $$d\vec{a}=dx\,dy\,\bold{\hat{z}}$$

        Combining the two, and drawing from the flux in the $\bold{\hat{z}}$ direction, we obtain:

        $$\int_0^2\int_0^2 (yz^2-3y)\,dx\,dy\Big|_{z=0}\Rightarrow\int_0^2\int_0^2 (3y)\,dx\,dy$$
        $$\boxed{2\int_0^2 (3y)\,dy=3y^2\Big|_0^2=12}$$

      \item Both surfaces have the same boundary. In this case, does the flux depend on the surface, or just the boundary? Explain.

    \end{enumerate}

  \item Consider a vector field $\vec{v}=r\cos(\theta)\bold{\hat{r}}+r\sin(\theta)\bold{\hat{\theta}}+r\sin(\theta)\cos(\theta)\bold{\hat{\phi}}$

    \begin{enumerate}

      \item Calculate the outward flux of $\vec{v}$ through a closed hemispherical surface of radius $R$ shown in the figure\footnote{figure omitted in this document}

        First and foremost, we know:

        $$d\vec{a}=\bold{\hat{r}}\,d\theta\,d\phi$$

        We also know the bounds of $\theta$ and $\phi$:

        $$\left\{\begin{array}{l} 0\leq\theta\leq2\pi\\0\leq\phi\leq\frac{\pi}{2}\end{array}$$

          This yields:

          $$\int_0^{\frac{\pi}{2}}\int_0^{2\pi} r\cos(\theta)\,d\theta\,d\phi\Big|_{r=R}\Rightarrow\frac{\pi R}{2}\int_0^{2\pi} \cos(\theta)\,d\theta=0$$

      \item Calculate the divergence of $\vec{v}$

        $$\frac{\partial}{\partial r}(r\cos(\theta))+\frac{\partial}{\partial \theta}(r\sin(\theta))+\frac{\partial}{\partial \phi}(r\sin(\theta)\cos(\theta))=\cos(\theta)+r\cos(\theta)$$

      \item Check the divergence theorem by comparing the flux integral from (a) with the volume integral of the divergence.

        For the hemisphere itself, we know the following:

        $$\left\{\begin{array}{l} 0\leq r\leq R\\0\leq\theta\leq 2\pi\\0\leq\phi\leq \frac{\pi}{2}\end{array}$$

        Then taking the boundaries defined above, we obtain the following integral expression:

        $$\int_0^{\frac{\pi}{2}}\int_0^{2\pi}\int_0^R(\cos(\theta)+r\cos(\theta))(r^2\sin(\theta))\,dr\,d\theta\,d\phi\Rightarrow $$
        $$\frac{\pi}{2}\int_0^{2\pi}\int_0^R(r^2\sin(\theta)\cos(\theta)+r^3\sin(\theta)\cos(\theta))\,dr\,d\theta\Rightarrow$$
        $$\frac{\pi}{2}\int_0^{2\pi}\frac{R^3}{3}\sin(\theta)\cos(\theta)+\frac{R^4}{4}\sin(\theta)\cos(\theta)\,d\theta\Rightarrow$$
        $$\frac{\pi}{2}\left( \frac{4R^3+3R^4}{12} \right)\underbrace{\int_0^{2\pi}\sin(\theta)\cos(\theta)\,d\theta}_{\text{0}}=0$$

        Then we need to find the flux through the bottom circle:

    \end{enumerate}

  \item 

    \begin{enumerate}

      \item 

      \item 

    \end{enumerate}

  \item Consider two vector functions $\vec{v}=x^2\bold{\hat{z}}$ and $\vec{w}=x\bold{\hat{x}}+y\bold{\hat{y}}+z\bold{\hat{z}}$

    \begin{enumerate}

      \item calculate the divergence and curl of each

        \begin{itemize}

          \item div$(x^2\bold{\hat{z}})$

            $$\boxed{\frac{\partial}{\partial x}(0)+\frac{\partial}{\partial y}(0)+\frac{\partial}{\partial z}(x^2)=0}$$

          \item curl$(x^2\bold{\hat{z}})$

            $$\vec{\nabla}\times\vec{v}=\left|\begin{matrix}\bold{\hat{x}} & \bold{\hat{y}} & \bold{\hat{z}}\\ \frac{\partial}{\partial x} & \frac{\partial}{\partial y} & \frac{\partial}{\partial z}\\ 0 & 0 & x^2\end{matrix}\right|=(0-0)\bold{\hat{x}}-(2x-0)\bold{\hat{y}}+(0-0)\bold{\hat{z}}=\boxed{-2x\bold{\hat{y}}}$$

          \item div$(\vec{w}=\vec{r})$

            $$\boxed{\frac{\partial}{\partial x}(x)+\frac{\partial}{\partial y}(y)+\frac{\partial}{\partial z}(z)=3}$$

          \item curl$(\vec{w}=\vec{r})$

            $$\vec{\nabla}\times\vec{v}=\left|\begin{matrix}\bold{\hat{x}} & \bold{\hat{y}} & \bold{\hat{z}}\\ \frac{\partial}{\partial x} & \frac{\partial}{\partial y} & \frac{\partial}{\partial z}\\ x & y & z\end{matrix}\right|=(0-0)\bold{\hat{x}}-(0-0)\bold{\hat{y}}+(0-0)\bold{\hat{z}}=\boxed{0}$$

        \end{itemize}

      \item 

      \item 

    \end{enumerate}

\end{enumerate}

\end{document}

