%%%%%%%%%%%%%%%%%%%%%%%%%%%%%%%%%%%%%%%%%%%%%%%%%%%%%%%%%%%%%%%%%%%%%%%%%%%%%%%%%%%%%%%%%%%%%%%%%%%%%%%%%%%%%%%%%%%%%%%%%%%%%%%%%%%%%%%%%%%%%%%%%%%%%%%%%%%%%%%%%%%
% Written By Michael Brodskiy
% Class: Electricity & Magnetism
% Professor: D. Wood
%%%%%%%%%%%%%%%%%%%%%%%%%%%%%%%%%%%%%%%%%%%%%%%%%%%%%%%%%%%%%%%%%%%%%%%%%%%%%%%%%%%%%%%%%%%%%%%%%%%%%%%%%%%%%%%%%%%%%%%%%%%%%%%%%%%%%%%%%%%%%%%%%%%%%%%%%%%%%%%%%%%

\documentclass[12pt]{article} 
\usepackage{alphalph}
\usepackage[utf8]{inputenc}
\usepackage[russian,english]{babel}
\usepackage{titling}
\usepackage{amsmath}
\usepackage{graphicx}
\usepackage{enumitem}
\usepackage{amssymb}
\usepackage[super]{nth}
\usepackage{everysel}
\usepackage{ragged2e}
\usepackage{geometry}
\usepackage{multicol}
\usepackage{fancyhdr}
\usepackage{cancel}
\usepackage{siunitx}
\usepackage{physics}
\usepackage{tikz}
\usepackage{mathdots}
\usepackage{yhmath}
\usepackage{cancel}
\usepackage{color}
\usepackage{array}
\usepackage{multirow}
\usepackage{gensymb}
\usepackage{tabularx}
\usepackage{extarrows}
\usepackage{booktabs}
\usepackage{lastpage}
\usetikzlibrary{fadings}
\usetikzlibrary{patterns}
\usetikzlibrary{shadows.blur}
\usetikzlibrary{shapes}

\geometry{top=1.0in,bottom=1.0in,left=1.0in,right=1.0in}
\newcommand{\subtitle}[1]{%
  \posttitle{%
    \par\end{center}
    \begin{center}\large#1\end{center}
    \vskip0.5em}%

}
\usepackage{hyperref}
\hypersetup{
colorlinks=true,
linkcolor=blue,
filecolor=magenta,      
urlcolor=blue,
citecolor=blue,
}


\title{Homework 1}
\date{\today}
\author{Michael Brodskiy\\ \small Professor: D. Wood}

\begin{document}

\maketitle

\begin{enumerate}
    
  \item Calculate:

    \begin{enumerate}

      \item $\vec{\nabla}\left( \frac{1}{r} \right)$

        We know $r=\sqrt{x^2+y^2+z^2}$. Converting and applying the chain rule, we get:

        $$\frac{\partial}{\partial x}\left( x^2+y^2+z^2\right)^{-\frac{1}{2}}\bold{\hat{x}}+\frac{\partial}{\partial y}\left( x^2+y^2+z^2\right)^{-\frac{1}{2}}\bold{\hat{y}}+\frac{\partial}{\partial z}\left( x^2+y^2+z^2\right)^{-\frac{1}{2}}\bold{\hat{z}}$$
        $$(x^2+y^2+z^2)^{-\frac{3}{2}}(x\bold{\hat{x}}+y\bold{\hat{y}}=z\bold{\hat{z}})$$
        $$\frac{x\bold{\hat{x}}+y\bold{\hat{y}}+z\bold{\hat{z}}}{r^3}$$

        We also know that $\vec{r}=x\bold{\hat{x}}+y\bold{\hat{y}}+z\bold{\hat{z}}$, and that $r\bold{\hat{r}}=\vec{r}$. Thus, we get:

        $$\boxed{\vec{\nabla}\left( \frac{1}{r} \right)=\frac{\vec{r}}{r^3}\rightarrow\frac{\bold{\hat{r}}}{r^2}}$$

      \item $\vec{\nabla}\cdot\bold{\hat{x}}$

        This implies the following:

        $$v_x=1\bold{\hat{x}},\quad v_y=0\bold{\hat{y}},\quad v_z=0\bold{\hat{z}}$$

        Thus, we find:

        $$\boxed{\vec{\nabla}\cdot\bold{\hat{x}}=\frac{\partial}{\partial x}(1\bold{\hat{x}})=0}$$

      \item $\vec{\nabla}\cdot\bold{\hat{r}}$

        This could be written as:

        $$\vec{\nabla}\cdot\left( \frac{x\bold{\hat{x}}+y\bold{\hat{y}}+z\bold{\hat{z}}}{\sqrt{x^2+y^2+z^2}} \right)$$

        Thus, we need to compute:

        $$\frac{\partial}{\partial x}\left(\frac{x}{\sqrt{x^2+y^2+z^2}}\right)$$

        with respect to each variable. By the quotient rule, we find

        $$\frac{\partial}{\partial x}\left(\frac{x}{\sqrt{x^2+y^2+z^2}}\right)\rightarrow \frac{\sqrt{x^2+y^2+z^2}-x^2(x^2+y^2+z^2)^{-0.5}}{x^2+y^2+z^2}$$

        Converting back to $r$, we obtain:

        $$\frac{1}{r}-\frac{x^2}{r^3}$$

        By symmetry, we know that the corresponding $y$ and $z$ variables become:

        $$\frac{1}{r}-\frac{y^2}{r^3}\quad\text{and}\quad\frac{1}{r}-\frac{z^2}{y^3}$$

        Summing the results, we get:

        $$\left( \frac{1}{r}-\frac{x^2}{r^3} \right)+\left( \frac{1}{r}-\frac{y^2}{r^3} \right)+\left( \frac{1}{r}-\frac{z^2}{r^3} \right)$$
        $$\frac{3}{r}-\left( \frac{x^2+y^2+z^2}{r^3} \right)$$
        $$\boxed{\vec{\nabla}\cdot\bold{\hat{r}}=\frac{3}{r}-\frac{1}{r}=\frac{2}{r}}$$

      \item $\vec{\nabla}r^n$ ($n>0$)

        $$\vec{\nabla}((x^2+y^2+z^2)^\frac{n}{2})\Rightarrow \frac{n}{2}(x^2+y^2+z^2)^{\frac{n}{2}-1}\left( 2x\bold{\hat{x}}+2y\bold{\hat{y}}+2z\bold{\hat{z}} \right)$$
        $$n\vec{r}r^{n-2}\rightarrow \frac{n\vec{r}}{r^{2-n}}$$

        Thus, we get:

        $$\boxed{\frac{n\bold{\hat{r}}}{r^{1-n}}\quad\text{or}\quad n\bold{\hat{r}}r^{n-1}}$$

    \end{enumerate}

  \item Calculate the divergence and curls of the following functions:

    \begin{enumerate}

      \item $\vec{v}_a=xy\bold{\hat{x}}+yz\bold{\hat{y}}+zy\bold{\hat{z}}$

        \begin{itemize}

          \item Divergence:

            $$\vec{\nabla}\cdot \vec{v}_a=\frac{\partial}{\partial x}(xy)+\frac{\partial}{\partial y}(yz)+\frac{\partial}{\partial z}(zy)$$
            $$\boxed{\text{div}(\vec{v}_a)=y+z+y=2y+z}$$

          \item Curl:

            $$\vec{\nabla}\times\vec{v}_a=\left|\begin{matrix} \bold{\hat{x}} & \bold{\hat{y}} & \bold{\hat{z}}\\ \frac{\partial}{\partial x} & \frac{\partial}{\partial y} & \frac{\partial}{\partial z} \\ xy & yz & zy \end{matrix}\right|=(z-y)\bold{\hat{x}}-(0-0)\bold{\hat{y}}+(0-x)\bold{\hat{z}}$$
            $$\boxed{\text{curl}(\vec{v}_a)=\langle z-y, 0, -x\rangle}$$

        \end{itemize}

      \item $\vec{v}_b=y^2\bold{\hat{x}}+(2xy+z^2)\bold{\hat{y}}+2xz\bold{\hat{z}}$

        \begin{itemize}

          \item Divergence:

            $$\vec{\nabla}\cdot\vec{v}_b=\frac{\partial}{\partial x}(y^2)+\frac{\partial}{\partial y}(2xy+z^2)+\frac{\partial}{\partial z}(2xz)$$
            $$\boxed{\text{div}(\vec{v}_b)=0+2x+2x=4x}$$

          \item Curl:

            $$\vec{\nabla}\times\vec{v}_b=\left|\begin{matrix} \bold{\hat{x}} & \bold{\hat{y}} & \bold{\hat{z}}\\ \frac{\partial}{\partial x} & \frac{\partial}{\partial y} & \frac{\partial}{\partial z} \\ y^2 & 2xy+z^2 & 2xz \end{matrix}\right|=(0-2z)\bold{\hat{x}}-(2z-0)\bold{\hat{y}}+(2y-2y)\bold{\hat{z}}$$
            $$\boxed{\text{curl}(\vec{v}_b)=\langle -2z, -2z, 0\rangle}$$

        \end{itemize}

      \item $\vec{v}_c=yz\bold{\hat{x}}+xz\bold{\hat{y}}+xy\bold{\hat{z}}$

        \begin{itemize}

          \item Divergence:

            $$\vec{\nabla}\cdot\vec{v}_c=\frac{\partial}{\partial x}(yz)+\frac{\partial}{\partial y}(xz)+\frac{\partial}{\partial z}(xy)$$
            $$\boxed{\text{div}(\vec{v}_c)=0+0+0=0}$$

          \item Curl:

            $$\vec{\nabla}\times\vec{v}_c=\left|\begin{matrix} \bold{\hat{x}} & \bold{\hat{y}} & \bold{\hat{z}}\\ \frac{\partial}{\partial x} & \frac{\partial}{\partial y} & \frac{\partial}{\partial z} \\ yz & xz & xy \end{matrix}\right|=(x-x)\bold{\hat{x}}-(y-y)\bold{\hat{y}}+(z-z)\bold{\hat{z}}$$
            $$\boxed{\text{curl}(\vec{v}_c)=\langle 0, 0, 0\rangle}$$

        \end{itemize}

    \end{enumerate}

  \item Calculate the following. Note that some results should be vectors.  It is helpful to check the dimensionality of your answers:

    \begin{enumerate}

      \item $(\vec{r}\cdot\vec{\nabla})\vec{r}$

        $$(\vec{r}\cdot\vec{\nabla})=\frac{\partial}{\partial x}(x\bold{\hat{x}})+\frac{\partial}{\partial y} (y\bold{\hat{y}})+\frac{\partial}{\partial z}(z\bold{\hat{z}})=1+1+1=3$$
        $$\boxed{(\vec{r}\cdot\vec{\nabla})\vec{r}=3\vec{r}=3x\bold{\hat{x}}+3y\bold{\hat{y}}+3z\bold{\hat{z}}}$$

      \item $(\bold{\hat{r}}\cdot\vec{\nabla})r$

        From Problem \textbf{1c}, we know the value of $(\bold{\hat{r}}\cdot\vec{\nabla})$, which gives us:

        $$\boxed{\left( \frac{2}{r} \right)r=2}$$

      \item $(\bold{\hat{r}}\cdot\vec{\nabla})\bold{\hat{r}}$

        Again, employing what we know from Problem \textbf{1c}, we get:

        $$\boxed{\left( \frac{2}{r} \right)\bold{\hat{r}}=\frac{2\bold{\hat{r}}}{r}}$$

    \end{enumerate}

  \item Consider a vector field $\vec{v}=2xz\bold{\hat{x}}+(x+2)\bold{\hat{y}}+y(z^2-3)\bold{\hat{z}}$ and a cube with vertices at (0,0,0), (0,2,0), (2,0,0), (0,0,2), etc. In the text (Example 1.7), the flux was calculated through the top five faces of the cube, and found to total to 20.

    \begin{enumerate}

      \item Find the upward flux through the bottom surface, i.e.\ the square surface in the $xy$-plane bounded by the points (0,0,0), (2,0,0), (2,2,0), (0,2,0).

        According to Gauss's Theorem, we know:

        $$\iiint \vec{\nabla}\cdot\vec{N}\,d\tau=\iint\vec{v}\cdot d\vec{a}$$

        Furthermore, because the surface is in the $xy$ plane, and the flux we want to find is in the $\bold{\hat{z}}$ direction, we know:

        $$d\vec{a}=dx\,dy\,\bold{\hat{z}}$$

        Combining the two, and drawing from the flux in the $\bold{\hat{z}}$ direction, we obtain:

        $$\int_0^2\int_0^2 (yz^2-3y)\,dx\,dy\Big|_{z=0}\Rightarrow\int_0^2\int_0^2 (3y)\,dx\,dy$$
        $$\boxed{2\int_0^2 (3y)\,dy=3y^2\Big|_0^2=12}$$

      \item Both surfaces have the same boundary. In this case, does the flux depend on the surface, or just the boundary? Explain.

        In this case, the flux depends on the surface, not just the boundary. If we take, for example, the top square — that is, the one bounded by $(0, 0, 2), (0, 2, 2), (2, 2, 2),$ and $(2, 0, 2)$, the integration would be consequently different. The boundary itself is the same; however, the flux is different. This occurs because $v_z$ is dependent on $z$; though the constraints on integration are the same, the following happens:

        $$\int_0^2\int_0^2(yz^2-3y)\,dx\,dy\Big|_{z=2}=\int_0^2\int_0^2(4y-3y)\,dx\,dy=4$$

        Thus, we see that, despite having the same boundary, the two surfaces yield a different flux.

    \end{enumerate}

  \item Consider a vector field $\vec{v}=r\cos(\theta)\bold{\hat{r}}+r\sin(\theta)\bold{\hat{\theta}}+r\sin(\theta)\cos(\theta)\bold{\hat{\phi}}$

    \begin{enumerate}

      \item Calculate the outward flux of $\vec{v}$ through a closed hemispherical surface of radius $R$ shown in the figure\footnote{figure omitted in this document}

        First and foremost, we know:

        $$d\vec{a}=\bold{\hat{r}}(r^2\sin(\theta))\,d\theta\,d\phi$$

        We also know the bounds of $\theta$ and $\phi$:

        $$\left\{\begin{array}{l} 0\leq\theta\leq\frac{\pi}{2}\\0\leq\phi\leq2\pi\end{array}$$

          This yields:

          $$\int_0^{2\pi}\int_0^{\frac{\pi}{2}} r^3\sin(\theta)\cos(\theta)\,d\theta\,d\phi\Big|_{r=R}\Rightarrow2\pi R^3\underbrace{\int_0^{\frac{\pi}{2}} \sin(\theta)\cos(\theta)\,d\theta}_{\frac{1}{2}}=\pi R^3$$

          Thus, the top part contributes a flux of $\pi R^3$. The bottom circle is constrained as follows:

          $$d\vec{a}=\bold{\hat{\theta}}r\,dr\,d\phi$$
          $$\left\{\begin{array}{l} \theta = \frac{\pi}{2}\\ 0\leq \phi\leq 2\pi\end{array}$$

            This yields:

            $$\int_0^{2\pi}\int_0^R r^2\sin(\theta)\,dr\,d\phi\Big|_{\theta=\frac{\pi}{2}}\Rightarrow 2\pi\int_0^Rr^2\,dr=2\pi \frac{R^3}{3}$$

            Summing the two together we get the flux as:

            $$\boxed{\frac{5\pi}{3}R^3}$$

      \item Calculate the divergence of $\vec{v}$

        $$\frac{1}{r^2}\frac{\partial}{\partial r}(r^3\cos(\theta))+\frac{1}{r\sin(\theta)}\frac{\partial}{\partial \theta}(r\sin^2(\theta))+\frac{1}{r\sin(\theta)}\frac{\partial}{\partial \phi}(r\sin(\theta)\cos(\theta))\Rightarrow$$
        $$3\cos(\theta)+2\cos(\theta)+0=\boxed{5\cos(\theta)}$$

      \item Check the divergence theorem by comparing the flux integral from (a) with the volume integral of the divergence.

        For the hemisphere itself, we know the following:

        $$\left\{\begin{array}{l} 0\leq r\leq R\\0\leq\theta\leq \frac{\pi}{2}\\0\leq\phi\leq 2\pi\end{array}$$

        Then taking the boundaries defined above, we obtain the following integral expression:

        $$\int_0^{2\pi}\int_0^{\frac{\pi}{2}}\int_0^R(5\cos(\theta))(r^2\sin(\theta))\,dr\,d\theta\,d\phi\Rightarrow $$
        $$2\pi\int_0^{\frac{\pi}{2}}\int_0^R(5r^2\cos(\theta)\sin(\theta))\,dr\,d\theta\Rightarrow$$
        $$\frac{10\pi R^3}{3}\int_0^{\frac{\pi}{2}}\sin(\theta)\cos(\theta)\,d\theta\Rightarrow$$
        $$\frac{1}{2}\left( \frac{10\pi R^3}{3} \right)=\boxed{\frac{5\pi}{3}R^3}$$

        Thus, the volume integral and flux integral are both $\frac{5\pi}{3}R^3$

    \end{enumerate}

  \item Start with these three expressions that relate rectangular coordinates to cylindrical coordinates: $x=s\cos(\phi), y=s\sin(\phi), z=z$

    \begin{enumerate}

      \item Derive the expressions for the unit vectors $\bold{\hat{s}},\bold{\hat{\phi}}, \bold{\hat{z}}$ in terms of $\bold{\hat{x}},\bold{\hat{y}},$ and $\bold{\hat{z}}$

        $$\bold{\hat{s}}=\frac{\frac{\partial}{\partial s}(s\cos(\phi))\bold{\hat{x}}+\frac{\partial}{\partial s}(s\sin(\phi))\bold{\hat{y}}+\frac{\partial}{\partial s}(z)\bold{\hat{z}}}{\sqrt{\cos^2(\phi)+\sin^2(\phi)}}=\cos(\phi)\bold{\hat{x}}+\sin(\phi)\bold{\hat{y}}$$
          $$\hat{\phi}=\frac{\frac{\partial}{\partial \phi}(s\cos(\phi))\bold{\hat{x}}+\frac{\partial}{\partial \phi}(s\sin(\phi))\bold{\hat{y}}+\frac{\partial}{\partial \phi}(z)\bold{\hat{z}}}{\sqrt{s^2\sin^2(\phi)+s^2\cos^2(\phi)}}=-\sin(\phi)\bold{\hat{x}}+\cos(\phi)\bold{\hat{y}}$$
          $$\bold{\hat{z}}=\frac{\frac{\partial}{\partial z}(s\cos(\phi))\bold{\hat{x}}+\frac{\partial}{\partial z}(s\sin(\phi))\bold{\hat{y}}+\frac{\partial}{\partial z}(z)\bold{\hat{z}}}{\sqrt{0^2+0^2+1^1}}=\bold{\hat{z}}$$

          Thus, we see the values as follows:

          $$\boxed{\left\{\begin{array}{l l l} \bold{\hat{s}} & = & \cos(\phi)\bold{\hat{x}}+\sin(\phi)\bold{\hat{y}}\\\bold{\hat{\phi}} & = & -\sin(\phi)\bold{\hat{x}}+\cos(\phi)\bold{\hat{y}}\\ \bold{\hat{z}} & = & \bold{\hat{z}}\end{array}}$$

          \item Confirm that $\bold{\hat{s}}, \bold{\hat{\phi}},$ and $\bold{\hat{z}}$ are mutually orthogonal and are normalized to unity\\

            We can check for orthogonality by applying the dot product; because $\bold{\hat{s}}$ and $\bold{\hat{\phi}}$ both only have components in the $\bold{\hat{x}}$ and $\bold{\hat{y}}$ directions, they are orthogonal to the $z$ component. As such, we need only check $\bold{\hat{s}}$ against $\bold{\hat{\phi}}$:

            $$\bold{\hat{s}}\cdot\bold{\hat{\phi}}=(\cos(\phi))(-\sin(\phi))+(\sin(\phi))(\cos(\phi))=0$$

            Thus, they are mutually orthogonal. To check for normalization, we must see whether the magnitude of each is equal to 1:

            $$\bold{\hat{s}}=\sqrt{\cos^2(\phi)+\sin^2(\phi)}=1$$
            $$\bold{\hat{\phi}}=\sqrt{\sin^2(\phi)+\cos^2(\phi)}=1$$
            $$\bold{\hat{z}}=\sqrt{1^2}=1$$

            $$\boxed{\text{As such, $\bold{\hat{s}}$, $\bold{\hat{\phi}}$, and $\bold{\hat{z}}$ are mutually orthogonal and normalized to unity}}$$

            In this manner, they make up a coordinate system.

    \end{enumerate}

  \item Consider two vector functions $\vec{v}=x^2\bold{\hat{z}}$ and $\vec{w}=x\bold{\hat{x}}+y\bold{\hat{y}}+z\bold{\hat{z}}$

    \begin{enumerate}

      \item calculate the divergence and curl of each

        \begin{itemize}

          \item div$(x^2\bold{\hat{z}})$

            $$\boxed{\frac{\partial}{\partial x}(0)+\frac{\partial}{\partial y}(0)+\frac{\partial}{\partial z}(x^2)=0}$$

          \item curl$(x^2\bold{\hat{z}})$

            $$\vec{\nabla}\times\vec{v}=\left|\begin{matrix}\bold{\hat{x}} & \bold{\hat{y}} & \bold{\hat{z}}\\ \frac{\partial}{\partial x} & \frac{\partial}{\partial y} & \frac{\partial}{\partial z}\\ 0 & 0 & x^2\end{matrix}\right|=(0-0)\bold{\hat{x}}-(2x-0)\bold{\hat{y}}+(0-0)\bold{\hat{z}}=\boxed{-2x\bold{\hat{y}}}$$

          \item div$(\vec{w}=\vec{r})$

            $$\boxed{\frac{\partial}{\partial x}(x)+\frac{\partial}{\partial y}(y)+\frac{\partial}{\partial z}(z)=3}$$

          \item curl$(\vec{w}=\vec{r})$

            $$\vec{\nabla}\times\vec{v}=\left|\begin{matrix}\bold{\hat{x}} & \bold{\hat{y}} & \bold{\hat{z}}\\ \frac{\partial}{\partial x} & \frac{\partial}{\partial y} & \frac{\partial}{\partial z}\\ x & y & z\end{matrix}\right|=(0-0)\bold{\hat{x}}-(0-0)\bold{\hat{y}}+(0-0)\bold{\hat{z}}=\boxed{0}$$

        \end{itemize}

      \item 

        \begin{itemize}

          \item For $\vec{v}=x^2\bold{\hat{z}}$:

            Assuming $v_y\rightarrow 0$, we obtain:

            $$\int x^2\,dz=x^2z+g(x)\Rightarrow \frac{\partial}{\partial x}(x^2z+g(x))=2xz+g'(x)=0$$
            $$g'(x)=-2xz\Rightarrow \int-2xz\,dx=-x^2z$$

            Thus, because this leaves us with simply $f(y)=0$, we see that a gradient representation is not possible.

          \item For $\vec{w}=\vec{r}$:

            $$\int x\,dx=\frac{x^2}{2}+f(y)+g(z)\Rightarrow \frac{\partial}{\partial y}\left( \frac{x^2}{2}+f(y)+g(z) \right)=f'(y)$$
            $$f'(y)=y\Rightarrow \int f'(y)\,dy=\frac{x^2}{2}+\frac{y^2}{2}+g(z)\Rightarrow\frac{\partial}{\partial z}\left(\frac{1}{2}\left( x^2+y^2+g(z) \right)\right)=g'(z)$$
            $$g'(z)=z\Rightarrow \int g'(z)\,dz=\frac{z^2}{2}+c$$

            Thus, the final expression for $g(x,y,z)$ becomes:

            $$\boxed{g_{\vec{w}}(x,y,z)=\frac{1}{2}\left( x^2+y^2+z^2 \right) + c}$$

        \end{itemize}

      \item 

    \end{enumerate}

\end{enumerate}

\end{document}

