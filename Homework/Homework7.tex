%%%%%%%%%%%%%%%%%%%%%%%%%%%%%%%%%%%%%%%%%%%%%%%%%%%%%%%%%%%%%%%%%%%%%%%%%%%%%%%%%%%%%%%%%%%%%%%%%%%%%%%%%%%%%%%%%%%%%%%%%%%%%%%%%%%%%%%%%%%%%%%%%%%%%%%%%%%%%%%%%%%
% Written By Michael Brodskiy
% Class: Electricity & Magnetism
% Professor: D. Wood
%%%%%%%%%%%%%%%%%%%%%%%%%%%%%%%%%%%%%%%%%%%%%%%%%%%%%%%%%%%%%%%%%%%%%%%%%%%%%%%%%%%%%%%%%%%%%%%%%%%%%%%%%%%%%%%%%%%%%%%%%%%%%%%%%%%%%%%%%%%%%%%%%%%%%%%%%%%%%%%%%%%

\include{Includes.tex}

\title{Homework 7}
\date{November 9, 2023}
\author{Michael Brodskiy\\ \small Professor: D. Wood}

\begin{document}

\maketitle

\begin{enumerate}

  \item A hollow sphere of radius $R$ centered at the origin is covered with a uniform surface charge $\sigma$ and is rotating about the $z$-axis with angular frequency $\omega$.  A uniform external magnetic field is oriented in the y-direction: $\vec{B}=B_o\bold{\hat{y}}$

    \begin{enumerate}

      \item Find the total force on the sphere

        Let us first define the dipole moment of the sphere. To do this, we must first define the current:

        $$I=\frac{1}{T}\left[ \sigma(2\pi R\sin(\theta)) R\,d\theta \right]$$
        $$I=\sigma\omega R^2\sin(\theta)\,d\theta$$

        This gives us the dipole moment as:

        $$d\vec{m}=IA\bold{\hat{z}}$$
        $$d\vec{m}=\sigma\omega R^2\sin(\theta)\,d\theta( \pi R^2\sin^2(\theta))\bold{\hat{z}}$$
        $$d\vec{m}=\pi\sigma\omega R^4\sin^3(\theta)\,d\theta\bold{\hat{z}}$$

        From here, we can find:

        $$\vec{m}=\pi\sigma\omega R^4\int\sin^3(\theta)\,d\theta\bold{\hat{z}}$$
        $$\vec{m}=\pi\sigma\omega R^4\int_0^\pi\sin^3(\theta)\,d\theta\bold{\hat{z}}$$
        $$\vec{m}=\frac{4}{3}\pi\sigma\omega R^4\bold{\hat{z}}$$

        Now, we can write the force as:

        $$\vec{F}=(\vec{m}\cdot\vec{\nabla})\vec{B}$$

        This gives us:

        $$\vec{F}=\left( \frac{4}{3}\pi\sigma\omega R^4\bold{\hat{z}}\cdot\vec{\nabla} \right)\vec{B}$$
        $$\vec{F}=\left( \frac{4}{3}\pi\sigma\omega R^4(\bold{\hat{z}}\cdot\vec{\nabla}) \right)\vec{B}$$
        $$\vec{F}=\left( \frac{4}{3}\pi\sigma\omega R^4 \right)\underbrace{\frac{\partial}{\partial z}(B_o\bold{\hat{y}})}_0$$

        Thus, we see that the net force is zero:

        $$\boxed{\vec{F}=0}$$

      \item Find the total torque on the sphere

        Using the dipole defined in (a), we can write:

        $$\vec{N}=\vec{m}\times\vec{B}$$
        $$\vec{N}=\left( \frac{4}{3}\pi\sigma\omega R^4\bold{\hat{z}}\right)\times(B_o\bold{\hat{y}})$$
        $$\vec{N}=\frac{4}{3}\pi\sigma\omega B_o R^4(\bold{\hat{z}}\times\bold{\hat{y}})$$
        $$\boxed{\vec{N}=-\frac{4}{3}\pi\sigma\omega B_o R^4\bold{\hat{x}}}$$

      \item Generalize the result of (b) to find the torque for a uniform magnetic field in an arbitrary direction, $\vec{B}=B_x\bold{\hat{x}}+B_y\bold{\hat{y}}+B_z\bold{\hat{z}}$

        Given the general case, we may write:

        $$\left|\begin{matrix} \bold{\hat{x}} & \bold{\hat{y}} & \bold{\hat{z}}\\ 0 & 0 & 1\\ B_x & B_y & B_z\end{matrix}\right|=(-B_y\bold{\hat{x}}+B_x\bold{\hat{y}})$$

        Thus, we can substitute $-B_o\bold{\hat{x}}$ from (b) with the above result to get:

        $$\boxed{\vec{N}=\frac{4}{3}\pi\sigma\omega R^4\left( B_x\bold{\hat{y}}-B_y\bold{\hat{x}} \right)}$$

    \end{enumerate}

  \item Calculate the magnetic field $\vec{B}(x,y)$ in the positive quadrant of the $x$-$y$ plane due to a current coming on the $y$-axis from $y=+\infty$, turning $90^{\circ}$ at the origin, and exiting along the $x$-axis to $x=+\infty$

    We know that the magnetic field at a point of distance $r$ from a wire can be defined as:

    $$\vec{B}=\frac{\mu_o I}{2\pi r}$$

    Thus, we can define the magnetic field produced by the $y$ current, with direction determined through the right hand rule, as:

    $$\vec{B}_y=\frac{\mu_o I}{2\pi y}\bold{\hat{z}}$$

    Using the same method, we can write the $x$ current magnetic field as:

    $$\vec{B}_x=\frac{\mu_o I}{2\pi x}\bold{\hat{z}}$$

    We can simply sum the magnetic field to find:

    $$\boxed{\vec{B}(x,y)=\frac{\mu_o I}{2\pi}\left( \frac{1}{x}+\frac{1}{y} \right)\bold{\hat{z}}}$$

  \item A long cylindrical conductor has a uniform current density $\vec{J}$ oriented along the axis.

    \begin{enumerate}

      \item Find the strength of the magnetic field as a function of $s$ (the perpendicular distance from the $z$-axis) for $s<a$.

        Using Amp\`ere's Law, we can write:

        $$\oint\vec{B}\cdot d\vec{l}=\mu_o I$$

        Applying the given parameters, we find:

        $$\oint\vec{B}\cdot d\vec{l}=\mu_o \vec{J}(\pi s^2)$$
        $$\vec{B}\oint d\vec{l}=\mu_o \vec{J}(\pi s^2)$$
        $$\vec{B}(2\pi s)=\mu_o \vec{J}(\pi s^2)$$
        $$\boxed{\vec{B}=\frac{\mu_o \vec{J} s}{2}}$$

      \item Consider a charged particle with charge $q$ and momentum $p$ that passes through the cylinder in part (a) with an initial velocity parallel to the axis.  As a function of the distance $s$ of the particle from the axis, find the angle of deflection after passing through a short distance $\Delta z$. Considered as lens for such charged particles, what it the focal length of this segment of the conductor as a function of $\vec{J},p,q,\Delta z$? (Such lenses are actually used in particle accelerators). Assume that $\Delta z<<R$, where $R$ is the radius of curvature of the particle in the magnetic field, so a small angle approximation is valid for the deflection

      \item Find the current density $\vec{J}$ needed to focus particles of charge $e=1.6\cdot10^{−19}[\si{\coulomb}]$ and momentum $p=75\left[ \frac{\si{\giga\eV}}{c} \right]$ with a focal length of $f=20 [\si{\meter}]$ for $\Delta z=0.50[\si{\meter}]$

    \end{enumerate}

  \item A double solenoid has two co-axial coils radii $a$ and $b$, with $n_a$ and $n_b$ turns per unit length, and with currents $I_a$ and $I_b$ flowing in opposite directions. Find:

    For the following problems, we can assume ideal solenoids, with current present only inside of the solenoid. The magnetic field may be defined as:

    $$\vec{B}=\mu_onI$$

    as was determined in class. Furthermore, let us define the axis of the solenoids as $\bold{\hat{x}}$, and assume that $I_a$ flows such that the magnetic field produced is in the $+\bold{\hat{x}}$ direction.

    \begin{enumerate}

      \item the magnetic field for the region inside the first coil ($s<a$)

        Inside the first coil, there are two currents, $I_a$ and $I_b$. Thus, we can simply sum the magnetic fields from each to determine:

        $$\boxed{\vec{B}_{s<a}=\mu_o\left( n_aI_a-n_bI_b \right)\bold{\hat{x}}}$$

      \item the magnetic field for the region between he two coils ($a<s<b$)

        Since $s$ is now outside of the first solenoid, the only field present is from the second solenoid. This gives us:

        $$\boxed{\vec{B}_{a<s<b}=-\mu_on_bI_b\bold{\hat{x}}}$$

      \item the magnetic field for the region inside outside both coils ($s>b$)

        Outside of both solenoids, there would be no field present. Thus, we may simply write:

        $$\boxed{\vec{B}_{s>b}=0}$$

      \item What ratio of currents would be required to have $\vec{B}=0$ for $s<a$?

        For the field to be zero, we find from (a):

        $$\mu_on_aI_a=\mu_on_bI_b$$

        Which results in a ratio of:

        $$\boxed{\frac{I_a}{I_b}=\frac{n_b}{n_a}}$$

    \end{enumerate}

\end{enumerate}

\end{document}

