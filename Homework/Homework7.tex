%%%%%%%%%%%%%%%%%%%%%%%%%%%%%%%%%%%%%%%%%%%%%%%%%%%%%%%%%%%%%%%%%%%%%%%%%%%%%%%%%%%%%%%%%%%%%%%%%%%%%%%%%%%%%%%%%%%%%%%%%%%%%%%%%%%%%%%%%%%%%%%%%%%%%%%%%%%%%%%%%%%
% Written By Michael Brodskiy
% Class: Electricity & Magnetism
% Professor: D. Wood
%%%%%%%%%%%%%%%%%%%%%%%%%%%%%%%%%%%%%%%%%%%%%%%%%%%%%%%%%%%%%%%%%%%%%%%%%%%%%%%%%%%%%%%%%%%%%%%%%%%%%%%%%%%%%%%%%%%%%%%%%%%%%%%%%%%%%%%%%%%%%%%%%%%%%%%%%%%%%%%%%%%

\include{Includes.tex}

\title{Homework 7}
\date{November 9, 2023}
\author{Michael Brodskiy\\ \small Professor: D. Wood}

\begin{document}

\maketitle

\begin{enumerate}

  \item A hollow sphere of radius $R$ centered at the origin is covered with a uniform surface charge $\sigma$ and is rotating about the $z$-axis with angular frequency $\omega$.  A uniform external magnetic field is oriented in the y-direction: $\vec{B}=B_o\bold{\hat{y}}$

    \begin{enumerate}

      \item Find the total force on the sphere

      \item Find the total torque on the sphere

      \item Generalize the result of (b) to find the torque for a uniform magnetic field in an arbitrary direction, $\vec{B}=B_x\bold{\hat{x}}+B_y\bold{\hat{y}}+B_z\bold{\hat{z}}$

    \end{enumerate}

  \item Calculate the magnetic field $\vec{B}(x,y)$ in the positive quadrant of the $x$-$y$ plane due to a current coming on the $y$-axis from $y=+\infty$, turning $90^{\circ}$ at the origin, and exiting along the $x$-axis to $x=+\infty$

  \item A long cylindrical conductor has a uniform current density $\vec{J}$ oriented along the axis.

    \begin{enumerate}

      \item Find the strength of the magnetic field as a function of $s$ (the perpendicular distance from the $z$-axis) for $s<a$.

      \item Consider a charged particle with charge $q$ and momentum $p$ that passes through the cylinder in part (a) with an initial velocity parallel to the axis.  As a function of the distance $s$ of the particle from the axis, find the angle of deflection after passing through a short distance $\Delta z$. Considered as lens for such charged particles, what it the focal length of this segment of the conductor as a function of $\vec{J},p,q,\Delta z$? (Such lenses are actually used in particle accelerators). Assume that $\Delta z<<R$, where $R$ is the radius of curvature of the particle in the magnetic field, so a small angle approximation is valid for the deflection

      \item Find the current density $\vec{J}$ needed to focus particles of charge $e=1.6\cdot10^{−19}[\si{\coulomb}]$ and momentum $p=75\left[ \frac{\si{\giga\eV}}{c} \right]$ with a focal length of $f=20 [\si{\meter}]$ for $\Delta z=0.50[\si{\meter}]$

    \end{enumerate}

  \item A double solenoid has two co-axial coils radii $a$ and $b$,with $n_a$ and $n_b$ turns per unit length, and with currents $I_a$ and $I_b$ flowing in opposite directions. Find:

    \begin{enumerate}

      \item the magnetic field for the region inside the first coil ($s<a$)

      \item the magnetic field for the region between he two coils ($a<s<b$)

      \item the magnetic field for the region inside outside both coils ($s>b$)

      \item What ratio of currents would be required to have $\vec{B}=0$ for $s<a$?

    \end{enumerate}

\end{enumerate}

\end{document}

