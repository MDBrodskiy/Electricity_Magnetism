%%%%%%%%%%%%%%%%%%%%%%%%%%%%%%%%%%%%%%%%%%%%%%%%%%%%%%%%%%%%%%%%%%%%%%%%%%%%%%%%%%%%%%%%%%%%%%%%%%%%%%%%%%%%%%%%%%%%%%%%%%%%%%%%%%%%%%%%%%%%%%%%%%%%%%%%%%%%%%%%%%%
% Written By Michael Brodskiy
% Class: Electricity & Magnetism
% Professor: D. Wood
%%%%%%%%%%%%%%%%%%%%%%%%%%%%%%%%%%%%%%%%%%%%%%%%%%%%%%%%%%%%%%%%%%%%%%%%%%%%%%%%%%%%%%%%%%%%%%%%%%%%%%%%%%%%%%%%%%%%%%%%%%%%%%%%%%%%%%%%%%%%%%%%%%%%%%%%%%%%%%%%%%%

\documentclass[12pt]{article} 
\usepackage{alphalph}
\usepackage[utf8]{inputenc}
\usepackage[russian,english]{babel}
\usepackage{titling}
\usepackage{amsmath}
\usepackage{graphicx}
\usepackage{enumitem}
\usepackage{amssymb}
\usepackage[super]{nth}
\usepackage{everysel}
\usepackage{ragged2e}
\usepackage{geometry}
\usepackage{multicol}
\usepackage{fancyhdr}
\usepackage{cancel}
\usepackage{siunitx}
\usepackage{physics}
\usepackage{tikz}
\usepackage{mathdots}
\usepackage{yhmath}
\usepackage{cancel}
\usepackage{color}
\usepackage{array}
\usepackage{multirow}
\usepackage{gensymb}
\usepackage{tabularx}
\usepackage{extarrows}
\usepackage{booktabs}
\usepackage{lastpage}
\usetikzlibrary{fadings}
\usetikzlibrary{patterns}
\usetikzlibrary{shadows.blur}
\usetikzlibrary{shapes}

\geometry{top=1.0in,bottom=1.0in,left=1.0in,right=1.0in}
\newcommand{\subtitle}[1]{%
  \posttitle{%
    \par\end{center}
    \begin{center}\large#1\end{center}
    \vskip0.5em}%

}
\usepackage{hyperref}
\hypersetup{
colorlinks=true,
linkcolor=blue,
filecolor=magenta,      
urlcolor=blue,
citecolor=blue,
}


\title{Homework 6}
\date{November 2, 2023}
\author{Michael Brodskiy\\ \small Professor: D. Wood}

\begin{document}

\maketitle

\begin{enumerate}

  \item A point charge of charge $q$ is located a distance $d$ from a neutral atom with polarizability $\alpha$.  The field from the point charge will induce a dipole moment in the atom, resulting in a force between the two objects.  Find the magnitude of the force and indicate if it is attractive or repulsive.

    We know that the electric field due to the point charge may be expressed as:

    $$\vec{E}_q=\frac{q}{4\pi\varepsilon_o r^2}\bold{\hat{r}}$$

    The induced dipole moment may be defined as:

    $$\vec{p}=\alpha\vec{E}_q=\frac{\alpha q}{4\pi\varepsilon_o r^2}\bold{\hat{r}}$$

    The horizontal electric field as a result of the dipole then becomes:

    $$\vec{E}_{\vec{p}}=\frac{2\vec{p}}{4\pi\varepsilon_o d^3}$$

    Since we know $d\to r$, we can simplify:

    $$\vec{E}_{\vec{p}}=\frac{1}{4\pi\varepsilon_o r^3}\left( 2\frac{\alpha q}{4\pi\varepsilon_o r^2 \right)}$$

    Since we know $\vec{F}=\vec{E}q$, we can find the force as:

    $$\vec{F}_{\vec{p}}=\frac{1}{4\pi\varepsilon_o r^3}\left( 2\frac{\alpha q}{4\pi\varepsilon_o r^2 \right)}q$$

    This can be simplified to:

    $$\boxed{\vec{F}_{\vec{p}}=\frac{2\alpha q^2}{(4\pi\varepsilon_o)^2r^5}}$$

    Because the sign opposes $q$, we know this is an attractive force

  \item A parallel plate capacitor has two metal plates and is filled with two different linear dielectrics. Each dielectric has thickness $d/2$.  The lower dielectric (A) has a dielectric constant $\varepsilon_r=3$ and the upper one (B) has dielectric constant $\varepsilon_r=5$.  The upper plate as a charge density of $+\sigma$ and the lower plate as a charge density of $-\sigma$

    \begin{enumerate}

      \item Find the electric displacement $\vec{D}$ in the dielectrics

        We can apply Gauss's law:

        $$\oint\vec{D}\,dA=q_f$$
        $$\vec{D}\oint\,dA=q_f$$
        $$\vec{D}A=\sigma A$$
        $$\vec{D}=\sigma$$

        This displacement is applied in the downward direction. Let us call this direction $\bold{\hat{x}}$. This is the same for both dielectrics:

        $$\boxed{\vec{D}_{A,B}=\sigma(-\bold{\hat{x}})}$$

      \item Find the electric field \vec{E} in each dielectric

        We can find the electric field according to the formula:

        $$\vec{D}=\varepsilon\vec{E}$$
        $$\vec{D}=\varepsilon_r\varepsilon_o\vec{E}$$

        This gives us:

        $$\boxed{\vec{E}_A=\frac{\sigma}{3\varepsilon_o}(-\bold{\hat{x}})}$$
        $$\boxed{\vec{E}_B=\frac{\sigma}{5\varepsilon_o}(-\bold{\hat{x}})}$$

      \item Find the potential difference between the plates

        The potential can be defined as:

        $$V=\vec{E}\cdot d$$

        Which we can then use to write:

        $$V=\left( \frac{\sigma}{3\varepsilon_o} +\frac{\sigma}{5\varepsilon_o}\right)\frac{d}{2}$$

        This evaluates to:

        $$\boxed{V=\frac{4\sigma d}{15\varepsilon_o}}$$

      \item Find the location and value of all of the bound charge

        We can begin by finding the polarizability:

        $$\vec{P}=\chi_e\varepsilon_o\vec{E}$$
        $$\vec{P}=\frac{\chi_e\vec{D}}{\varepsilon_r}$$
        $$\vec{P}=\frac{(\varepsilon_r-1)\sigma}{\varepsilon_r}(-\bold{\hat{x}})$$

        Thus, we find:

        $$\vec{P}_A=\frac{2\sigma}{3}(-\bold{\hat{x}})$$
        $$\vec{P}_B=\frac{4\sigma}{5}(-\bold{\hat{x}})$$

        We can see that $\rho_b=0$, since $\vec{P}$ is constant. Thus, we need only to find the surface charge, $\sigma_b$, which we can do by dotting with the normal vector. There are four segments for which we need to find values. These are the top and bottom of $B$ and $A$, which are:


        $$\boxed{\sigma_{\uparrow B}=-\frac{4\sigma}{5}}$$
        $$\boxed{\sigma_{\downarrow B}=\frac{4\sigma}{5}}$$
        $$\boxed{\sigma_{\uparrow A}=-\frac{2\sigma}{3}}$$
        $$\boxed{\sigma_{\downarrow A}=\frac{2\sigma}{3}}$$

    \end{enumerate}

  \item Calculate the minimum possible volume for a $1[\si{\farad}]$ capacitor that can withstand $2.5[\si{\volt}]$ without breaking down. Assume that the geometry and plate separation can be optimized and that the thickness of the conducting plates is negligible.

    \begin{enumerate}

      \item Assume the dielectric is air (dielectric strength = $3[\si{\mega\volt}/\si{\meter}]$, dielectric constant = 1)

        In this case, we know $V=2.5[\si{\volt}]$, $\vec{E}=3\cdot10^6[\si{\volt}/\si{\meter}]$, $C=1[\si{\farad}]$, and $\varepsilon_r=0$. We can then write:

        $$\vec{E}=\frac{V}{d}$$
        $$d=\frac{V}{\vec{E}}$$
        $$d=\frac{2.5}{3\cdot10^{6}}$$
        $$d=8.33\bar{3}\cdot10^{-7}[\si{\meter}]$$

        We know:

        $$C=\frac{\varepsilon_o\varepsilon_r A}{d}$$
        $$A=\frac{d C}{\varepsilon_o\varepsilon_r}$$
        $$A=\frac{8.33\bar{3}\cdot10^{-7}(1)}{8.85\cdot10^{-12}}$$
        $$A=9.416\cdot10^4[\si{\meter\squared}]$$

        Now, we can find the volume:

        $$V=A\cdot d$$
        $$V=9.416\cdot10^4\cdot 8.33\bar{3}\cdot10^{-7}$$
        $$\boxed{V=.0785[\si{\meter\cubed}]}$$

      \item Assume the dielectric is strontium titanate (dielectric strength = $8[\si{\mega\volt}/\si{\meter}]$, dielectric constant = 233)

        We can use a similar approach as that from (a):

        $$d=\frac{V}{\vec{E}}$$
        $$d=\frac{2.5}{8\cdot10^6}$$
        $$d=3.125\cdot10^{-7}[\si{\meter}]$$

        And then:

        $$A=\frac{Cd}{\varepsilon_o\varepsilon_r}$$
        $$A=\frac{(1)(3.125\cdot10^{-7})}{8.85\cdot10^{-12}\cdot233}$$
        $$A=151.55[\si{\meter\squared}]$$

        Finally, we get:

        $$V=Ad$$
        $$V=\left( 3.125\cdot10^{-7} \right)\left( 151.55 \right)$$
        $$\boxed{V=4.736\cdot10^{-5}[\si{\meter\cubed}]}$$

      Hint: Consider stored energy per unit volume

    \end{enumerate}

  \item Two long coaxial cylindrical metal tubes (inner radius $a$ and outer radius $b$) stand vertically in a tank of dielectric oil (susceptibility $\chi_e$, mass density $\rho_m$).  The inner cylinder is maintained at a potential $V$ and the outer one is grounded. To what height $h$ does the oil rise, in the space between the tubes?

    We know that for a long coaxial cable, with charge density $\lambda$, the electric field can be defined as:

    $$\vec{E}=\frac{\lambda}{2\pi\varepsilon_o r}\bold{\hat{r}}$$

    Thus, for the air portion, the voltage would be:

    $$V=\int_a^b \frac{\lambda}{2\pi\varepsilon_o r}\,dr$$
    $$V(r)=\frac{\lambda}{2\pi\varepsilon_o}\ln\left( \frac{b}{a} \right)$$

    For the oil portion, we can write:

    $$\vec{E}=\frac{\lambda\prime}{2\pi\varepsilon r}\bold{\hat{r}}$$

    Which then gives:

    $$V(r)=\frac{\lambda\prime}{2\pi\varepsilon}\ln\left( \frac{b}{a} \right)$$

    We then set the two voltages equal to obtain:

    $$\lambda\prime=\frac{\lambda\varepsilon}{\varepsilon_o}$$

    We know the total charge in the cable with total length $l$ is:

    $$Q=\lambda(l-h)+\lambda\prime h$$

    We can substitute our value of $\lambda\prime$ to get:

    $$Q=\lambda(l-h)+\frac{\lambda\varepsilon h}{\varepsilon_o}$$

    Since we know $\varepsilon=\varepsilon_o\varepsilon_r$, we can write:

    $$Q=\lambda(l-h)+\lambda\varepsilon_r h$$
    $$Q=\lambda l+\lambda h(\varepsilon_r-1)$$

    Furthermore, since $(\varepsilon_r-1)=\chi_e$, we get:

    $$Q=\lambda (l+h\chi_e)$$

    We then use the definition of capacitance to find:

    $$C=\frac{Q}{\Delta V}$$

    This gives us:

    $$C=\frac{\lambda(l+h\chi_e)(2\pi\varepsilon_o)}{\lambda\ln\left( \frac{b}{a} \right)}$$
    $$C=\frac{(l+h\chi_e)(2\pi\varepsilon_o)}{\ln\left( \frac{b}{a} \right)}$$

    Since we know the energy of a capacitor is defined as:

    $$U=\frac{1}{2}CV^2$$

    We can find the upward force using:

    $$F=\frac{1}{2}V^2\frac{\partial C}{\partial h}$$

    This is defined as:

    $$\frac{\partial}{\partial h}\left( \frac{(l+h\chi_e)(2\pi\varepsilon_o)}{\ln\left( \frac{b}{a} \right)} \right)\to\frac{2\chi_e\pi\varepsilon_o}{\ln\left( \frac{b}{a} \right)}$$

    This gives us an upward force of:

    $$F=\frac{\chi_e\pi\varepsilon_oV^2}{\ln\left( \frac{b}{a} \right)}$$

    Now, we need to find the gravitational force (which acts downward) on the oil. This can be done using the density of the oil:

    $$m=\rho_mV$$

    The volume of the area with oil present may be defined as the difference of volumes with radii $a$ and $b$:

    $$V=\pi(b^2-a^2)h$$
    $$m=\rho_m\pi\left( b^2-a^2 \right)h$$

    The gravitational force is known to be $F=mg$, which gives us:

    $$F=\rho_m\pi\left( b^2-a^2 \right)hg$$

    We want the oil to be in equilibrium, so the upward force must be equal to the downward force. This yields:

    $$\rho_m\pi\left( b^2-a^2 \right)hg=\frac{\chi_e\pi\varepsilon_oV^2}{\ln\left( \frac{b}{a} \right)}$$

    We then rearrange to obtain the height as:

    $$\boxed{h=\frac{\chi_e\varepsilon_o V^2}{g\rho_m(b^2-a^2)\ln\left( \frac{b}{a} \right)}}$$

  \item A point charge with charge $q$ is fixed at the center of a sphere of radius $R$ made of a linear dielectric material with susceptibility $\chi_e$. Find:

    \begin{enumerate}

      \item The electric field outside the sphere

        The electric field outside the sphere can be defined as normal, using Gauss's law:

        $$\oint \vec{E}\cdot d\vec{a}=\frac{q_{enc}}{\varepsilon_o}$$
        $$\vec{E}\oint d\vec{a}=\frac{q_{enc}}{\varepsilon_o}$$
        $$\boxed{\vec{E}_{r>R}=\frac{q}{4\pi\varepsilon_o r^2}\bold{\hat{r}}}$$

      \item The electric field in the sphere

        Within the material, we need to define the electric field with respect to $\vec{D}$:

        $$\oint \vec{D}\cdot d\vec{a}=q_{enc}$$
        $$\vec{D}\oint d\vec{a}=q_{enc}$$
        $$\vec{D}=\frac{q}{4\pi r^2}\bold{\hat{r}}$$

        Furthermore, we know $\vec{D}=\varepsilon_r\vec{E}=\varepsilon_o(1+\chi_e)\vec{E}$, which yields:

        $$\boxed{\vec{E}_{r<R}=\frac{q}{4\pi\varepsilon_o(1+\chi_e)r^2}\bold{\hat{r}}}$$

      \item The bound volume charged density $\rho_b$

        The polarization may be defined as:

        $$\vec{P}=\varepsilon_o\chi_e\vec{E}_{r<R}$$

        We can find the bound volume charge density by taking the negative gradient of the polarization:

        $$-\vec{\nabla}(\vec{P})=\vec{\nabla}\left( \frac{q\chi_e}{4\pi(1+\chi_e)r^2}\bold{\hat{r}} \right)$$
        $$\rho_b=-\frac{q\chi_e}{4\pi(1+\chi_e)}\vec{\nabla}\left( \frac{\bold{\hat{r}}}{r^2} \right)$$

        We know:

        $$\vec{\nabla}\left( \frac{\bold{\hat{r}}}{r^2} \right)=4\pi\delta^2(\vec{r})$$

        Thus, we end up with:

        $$\boxed{\rho_b=-\frac{q\chi_e}{(1+\chi_e)}\delta^3(\vec{r})}$$

      \item The bound surface charge density $\sigma_b$ on the outer surface

        The surface charge density can be found once again using the polarization, except with $r=R$:

        $$\sigma_b=\vec{P}_{r=R}\cdot\bold{\hat{r}}$$

        This gives us:

        $$\boxed{\sigma_b=\frac{q\chi_e}{4\pi(1+\chi_e)R^2}}$$

        \noindent\fbox{%
          \parbox{.85\textwidth}{%
              Interesting question (not for credit): The dielectric sphere itself must be neutral, so where is the missing charge?
            }%
        }\\

        Using our result from part (d), we can see:

        $$q_{surf}=\sigma_b\cdot A=\sigma_b(4\pi R^2)$$
        $$q_{surf}=\frac{q\chi_e}{(1+\chi_e)}$$

        Thus, we see that, to compensate, there must be charge located at the center.

    \end{enumerate}

\end{enumerate}

\end{document}

