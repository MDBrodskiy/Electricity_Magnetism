%%%%%%%%%%%%%%%%%%%%%%%%%%%%%%%%%%%%%%%%%%%%%%%%%%%%%%%%%%%%%%%%%%%%%%%%%%%%%%%%%%%%%%%%%%%%%%%%%%%%%%%%%%%%%%%%%%%%%%%%%%%%%%%%%%%%%%%%%%%%%%%%%%%%%%%%%%%%%%%%%%%
% Written By Michael Brodskiy
% Class: Electricity & Magnetism
% Professor: D. Wood
%%%%%%%%%%%%%%%%%%%%%%%%%%%%%%%%%%%%%%%%%%%%%%%%%%%%%%%%%%%%%%%%%%%%%%%%%%%%%%%%%%%%%%%%%%%%%%%%%%%%%%%%%%%%%%%%%%%%%%%%%%%%%%%%%%%%%%%%%%%%%%%%%%%%%%%%%%%%%%%%%%%

\include{Includes.tex}

\title{Homework 6}
\date{November 2, 2023}
\author{Michael Brodskiy\\ \small Professor: D. Wood}

\begin{document}

\maketitle

\begin{enumerate}

  \item A point charge of charge $q$ is located a distance $d$ from a neutral atom with polarizability $\alpha$.  The field from the point charge will induce a dipole moment in the atom, resulting in a force between the two objects.  Find the magnitude of the force and indicate if it is attractive or repulsive.

  \item A parallel plate capacitor has two metal plates and is filled with two different linear dielectrics. Each dielectric has thickness $d/2$.  The lower dielectric (A) has a dielectric constant $\varepsilon_r=3$ and the upper one (B) has dielectric constant $\varepsilon_r=5$.  The upper plate as a charge density of $+\sigma$ and the lower plate as a charge density of $-\sigma$

    \begin{enumerate}

      \item Find the electric displacement $\vec{D}$ in the dielectrics

      \item Find the electric field $\vec{E}$ in each dielectric

      \item Find the potential difference between the plates

      \item Find the location and value of all of the bound charge

    \end{enumerate}

  \item Calculate the minimum possible volume for a $1[\si{\farad}]$ capacitor that can withstand $2.5[\si{\volt}]$ without breaking down. Assume that the geometry and plate separation can be optimized and that the thickness of the conducting plates is negligible.

    \begin{enumerate}

      \item Assume the dielectric is air (dielectric strength = $3[\si{\mega\volt}/\si{\meter}]$, dielectric constant = 1)

      \item Assume the dielectric is strontium titanate (dielectric strength = $8[\si{\mega\volt}/\si{\meter}]$, dielectric constant = 233)

      Hint: Consider stored energy per unit volume

    \end{enumerate}

  \item Two long coaxial cylindrical metal tubes (inner radius $a$ and outer radius $b$) stand vertically in a tank of dielectric oil (susceptibility $\chi_e$, mass density $\rho_m$).  The inner cylinder is maintained at a potential $V$ and the outer one is grounded. To what height $h$ does the oil rise, in the space between the tubes?

  \item A point charge with charge $q$ is fixed at the center of a sphere of radius $R$ made of a linear dielectric material with susceptibility $\chi_e$. Find:

    \begin{enumerate}

      \item The electric field outside the sphere

      \item The electric field in the sphere

      \item The bound volume charged density $\rho_b$

      \item The bound surface charge density $\sigma_b$ on the outer surface

      Interesting question (not for credit): The dielectric sphere itself must be neutral, so where is the missing charge?

    \end{enumerate}

\end{enumerate}

\end{document}

