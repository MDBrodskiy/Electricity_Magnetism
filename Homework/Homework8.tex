%%%%%%%%%%%%%%%%%%%%%%%%%%%%%%%%%%%%%%%%%%%%%%%%%%%%%%%%%%%%%%%%%%%%%%%%%%%%%%%%%%%%%%%%%%%%%%%%%%%%%%%%%%%%%%%%%%%%%%%%%%%%%%%%%%%%%%%%%%%%%%%%%%%%%%%%%%%%%%%%%%%
% Written By Michael Brodskiy
% Class: Electricity & Magnetism
% Professor: D. Wood
%%%%%%%%%%%%%%%%%%%%%%%%%%%%%%%%%%%%%%%%%%%%%%%%%%%%%%%%%%%%%%%%%%%%%%%%%%%%%%%%%%%%%%%%%%%%%%%%%%%%%%%%%%%%%%%%%%%%%%%%%%%%%%%%%%%%%%%%%%%%%%%%%%%%%%%%%%%%%%%%%%%

\include{Includes.tex}

\title{Homework 8}
\date{November 17, 2023}
\author{Michael Brodskiy\\ \small Professor: D. Wood}

\begin{document}

\maketitle

\begin{enumerate}

  \item We revisit the spinning hollow sphere from homework 7. This time we consider the magnetic field produced by the sphere itself.  A hollow insulating sphere of radius $R$ centered at the origin is covered with a uniform surface charge $\sigma$ and is rotating about the $z$-axis with angular frequency $\omega$.

    \begin{enumerate}

      \item Compute the magnetic moment of the sphere. It might be helpful to model is as a collection of rings of current.

      \item Find the vector potential outside of the sphere due to this magnetic dipole moment (just the dipole term)Compare the result with the calculation of the full vector potential from the text (example 5.11, eqn. 5.69).  What can you conclude about the higher magnetic moments (quadrupole and higher)?

      \item Compute the magnetic field from the vector potential found in (b).

    \end{enumerate}

  \item Consider a square loop of current in the $xy$-plane with sides of length $L$ with its center at the origin.

    \begin{enumerate}

      \item Find the magnetic field on the center axis ($z$-axis) as a function of $z$

      \item Find the approximate field for $z>>L$.  Show that this corresponds to a dipole field, and determine the corresponding dipole moment.

    \end{enumerate}

  \item Griffiths 5.58 (5.56 in \nth{3} Edition): A thin uniform annulus (donut) carrying uniform charge $Q$ and mass $M$ rotates about its central axis.

    \begin{enumerate}

      \item Find the ratio of its magnetic dipole moment to its angular momentum, a.k.a.\ the gyromagnetic ratio, in terms of $Q$, $M$, and any fundamental constants.

      \item What is this ratio for a uniform spinning sphere?  Do not do a new calculation, but think of the sphere a made up of many rings and apply the result from (a)

      \item From quantum mechanics, we know that the angular momentum of the electron along an axis is $\hbar/2$, where $\hbar=h/(2\pi)$ is the reduced Planck’s constant. What is the semiclassical prediction from (b) for the magnetic moment of the electron along that axis? [The full relativistic quantum field theory calculation is large by almost exactly a factor of 2.  See this problem in Griffiths for of the significance and history of the gyromagnetic ratio of this calculation]

    \end{enumerate}

  \item A long cylinder of radius $R$ with its axis along the $z$-axis has a magnetization given by $M=ks^\hat{\phi}$. Find the magnetic field $\vec{B}$ for 

    \begin{enumerate}

      \item $s<R$

      \item $s>R$

    \end{enumerate}

  \item A toroid with 100 turns has an inner radius of 10 cm and an outer radius of 20 cm. It is filled with Gadolinium ($\chi_e=0.48$). Find the magnetic field inside the toroid at a distance of 15 cm from the center when a current of 100 A is flowing through the coils.

\end{enumerate}

\end{document}

