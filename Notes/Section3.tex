%%%%%%%%%%%%%%%%%%%%%%%%%%%%%%%%%%%%%%%%%%%%%%%%%%%%%%%%%%%%%%%%%%%%%%%%%%%%%%%%%%%%%%%%%%%%%%%%%%%%%%%%%%%%%%%%%%%%%%%%%%%%%%%%%%%%%%%%%%%%%%%%%%%%%%%%%%%%%%%%%%%
% Written By Michael Brodskiy
% Class: Electricity & Magnetism
% Professor: D. Wood
%%%%%%%%%%%%%%%%%%%%%%%%%%%%%%%%%%%%%%%%%%%%%%%%%%%%%%%%%%%%%%%%%%%%%%%%%%%%%%%%%%%%%%%%%%%%%%%%%%%%%%%%%%%%%%%%%%%%%%%%%%%%%%%%%%%%%%%%%%%%%%%%%%%%%%%%%%%%%%%%%%%

\documentclass[12pt]{article} 
\usepackage{alphalph}
\usepackage[utf8]{inputenc}
\usepackage[russian,english]{babel}
\usepackage{titling}
\usepackage{amsmath}
\usepackage{graphicx}
\usepackage{enumitem}
\usepackage{amssymb}
\usepackage[super]{nth}
\usepackage{everysel}
\usepackage{ragged2e}
\usepackage{geometry}
\usepackage{multicol}
\usepackage{fancyhdr}
\usepackage{cancel}
\usepackage{siunitx}
\usepackage{physics}
\usepackage{tikz}
\usepackage{mathdots}
\usepackage{yhmath}
\usepackage{cancel}
\usepackage{color}
\usepackage{array}
\usepackage{multirow}
\usepackage{gensymb}
\usepackage{tabularx}
\usepackage{extarrows}
\usepackage{booktabs}
\usepackage{lastpage}
\usetikzlibrary{fadings}
\usetikzlibrary{patterns}
\usetikzlibrary{shadows.blur}
\usetikzlibrary{shapes}

\geometry{top=1.0in,bottom=1.0in,left=1.0in,right=1.0in}
\newcommand{\subtitle}[1]{%
  \posttitle{%
    \par\end{center}
    \begin{center}\large#1\end{center}
    \vskip0.5em}%

}
\usepackage{hyperref}
\hypersetup{
colorlinks=true,
linkcolor=blue,
filecolor=magenta,      
urlcolor=blue,
citecolor=blue,
}


\title{Potentials}
\date{\today}
\author{Michael Brodskiy\\ \small Professor: D. Wood}

\begin{document}

\maketitle

\begin{itemize}

    $$\oint\vec{E}\cdot d\vec{a}=\frac{q_{enc}}{\varepsilon_o}\Leftrightarrow \vec{\nabla}\cdot\vec{E}=\frac{\rho}{\varepsilon_o}$$

  \item This means that, in a region with no charge, $\vec{\nabla}\cdot\vec{E}=0$, which also means $\nabla^2V=0$ (the Laplacian)

  \item Rectangular Coordinates:

    $$\nabla^2V=0\quad\quad \frac{\partial^2V}{\partial x^2}+\frac{\partial^2V}{\partial y^2}+\frac{\partial^2V}{\partial z^2}=0$$

    $$V(x,y,z)=X(x)Y(y)Z(z)\Rightarrow \frac{\partial^2X}{\partial x^2}YZ+\frac{\partial^2Y}{\partial y^2}XZ+\frac{\partial^2Z}{\partial z^2}XY=0$$

  \item If we divide by the respective functions, we get:

    $$\frac{\partial^2}{\partial x^2}\frac{1}{X}+\frac{\partial^2}{\partial y^2}\frac{1}{Y}+\frac{\partial^2}{\partial z^2}\frac{1}{Z}=0$$
    $$\frac{d^2X}{dx^2}=c_1X(x)\Rightarrow c_1>0:$$
    $$X(x)=\left\{\begin{array}{l r} Ae^{\pm kx} & c_1=k^2\\A\sin(kx) & c_1=-k^2\\A\cos(kx) & c_1=-k^2\end{array}$$

      \begin{itemize}

        \item Repeating this for each variable, we find

          $$\frac{d^2Y}{dy^2}=c_2Y(y)\quad\quad\frac{d^2Z}{dz^2}=c_2Z(z)$$
          $$c_1+c_2+c_3=0$$

      \end{itemize}

    \item Fourier Inversion

      $$\int_0^a\sin\left( \frac{n\pi y}{a} \right)\sin\left( \frac{m\pi y}{a} \right)\,dy=\frac{a}{2}\delta_{nm}$$

      This is used to define the terms for Fourier analysis

    \item A semi-infinite square tube, with the four walls grounded:

      \begin{itemize}

        \item The boundary conditions are (with $V=0$) given by:

          $$\left\{\begin{array}{l l} x=0, & 0<y<b\\x=a, & 0<y<b\\ y=0, & 0<x<a\\y=b, & 0<x<a\end{array}$$

          \item This would mean:

            $$X=\sin\left( \frac{n\pi x}{a} \right)\quad\quad Y\propto\sin\left( \frac{m\pi y}{b} \right)\quad\quad Z=e^{-k_zz}$$
            $$V=\sum B_{n,m}e^{-k_zz}\sin\left( \frac{n\pi x}{a} \right)\sin\left( \frac{m\pi y}{b} \right)$$

          \item Because we know $c_1+c_2+c_3=0$, we know:

            $$-\left( \frac{n\pi}{a} \right)^2-\left( \frac{m\pi}{b} \right)+k_z^2=0$$
            $$k_z=\sqrt{\left( \frac{n\pi}{a} \right)^2+\left( \frac{m\pi}{b} \right)^2}$$

      \end{itemize}

    \item Poisson's Equation

      $$\nabla^2V=-\frac{\rho}{\varepsilon_o}$$

      \begin{itemize}

        \item When the charge density is zero, we get Laplace's Equation:

          $$\nabla^2V=0$$

      \end{itemize}

    \item Given a box in three dimensions with lengths $a,b,c$, we know:

      $$c_1+c_2+c_3=0\longrightarrow\left\{\begin{array}{c c}c>0, & \text{Exponential}\\c<0, & \text{Sinusoidal}\end{array}$$

      \begin{itemize}

        \item We know the voltage is of the form

          $$V=f(z)\sin\left( \frac{n\pi x}{a} \right)\sin\left( \frac{m\pi y}{a} \right)$$

        \item The function, $f(z)$, must then be of the form:

          $$f(z)=Ae^{kz}+Be^{-kz}$$

        \item If $V=0$, then $z=0$, so:

          $$A+B=0\rightarrow A=-B$$
          $$f(z)=Ae^{kz}-Ae^{-kz}$$
          $$=A(e^{kz}-e^{-kz})$$
          $$=2A\sinh(kz)$$

        \item Thus, we can set up the function as:

          $$V(x,y,z)=\sum_{m,n} c_{m,n}\sin\left( \frac{n\pi x}{a} \right)\sin\left( \frac{m\pi y}{b} \right)\sinh(kz)$$

      \end{itemize}

    \item Spherical Coordinates

      $$\nabla^2V=\frac{1}{r^2}\frac{\partial}{\partial r}\left( r^2\frac{\partial V}{\partial r^2} \right)+\frac{1}{r^2\sin(\theta)}\frac{\partial}{\partial \theta}\left( \sin(\theta)\frac{\partial V}{\partial \theta} \right)=0$$

      \begin{itemize}

        \item In Quantum Mechanics, we have spherical harmonics:

          $$Y_{lm}(\theta,\phi)=P_l(\cos(\theta))f_{ml}(\phi)$$

        \item For our purposes, we can find:

          $$V(r,\theta)=\sum_{l=0,1,\ldots}\left( A_lr^l+\frac{B_l}{r^{l+1}} \right)P_e(\cos(\theta))$$

        \item A typical problem for this would be:

          \begin{itemize}

            \item Given some potential, $V(\theta)$ on a spherical surface, find $V_{in}$ and $V_{out}$

              $$R_e(r)=A_lr^l+\frac{B_l}{r^{l+1}}$$
              $$V_{in}=A_lr^lP_l(\cos(\theta))$$
              $$V_{out}=\frac{B_l}{r^{l+1}}P_l(\cos(\theta))$$

            \item We know the following boundary conditions:

              $$r\to\infty,V\to0$$

            \item We can integrate the function:

              $$\int_{-1}^1 V(R,\theta)P_l(\cos(\theta))d(\cos(\theta))=\frac{2R^lA_l}{2l+1}$$

          \end{itemize}

        \item Example: Given $V(R,\theta)=E_oR\cos(\theta)=E_oRP_1(\cos(\theta))$

          \begin{itemize}

            \item This gives us:

              $$\left\{\begin{array}{l l}\text{Inside: } & V=A_1r^2\cos(\theta)\\\text{Outside: } & V=\frac{B_1}{r^2}P_1(\cos(\theta))\end{array}$$

          \end{itemize}

      \end{itemize}

    \item For a sphere

      \begin{itemize}

        \item We find $E_oR\cos(\theta)$:

          $$\left\{\begin{array}{l l}\text{Inside: } & V=E_or\cos(\theta)=E_oz\\\text{Outside: } & V=\dfrac{E_oR^3}{r^2}(\cos(\theta))\end{array}$$

        \item We know:

          $$\vec{E}_{in}=-\vec{\nabla}V_{in}=-\frac{\partial V}{\partial z}\bold{\hat{z}}=-E_o\bold{\hat{z}}=E_o(-\cos(\theta)\bold{\hat{r}}+\sin(\theta)\hat{\theta})$$
          $$\vec{E}_{out}=-\vec{\nabla}V_{out}=-\frac{\partial V}{\partial r}\bold{\hat{r}}=-\frac{1}{r}\frac{\partial V_{out}}{\partial\theta}=\frac{E_oR^3}{r^3}(2\cos(\theta)\bold{\hat{r}}+\sin(\theta)\hat{\theta})$$

        \item To confirm our calculation, we can take the partial derivative with respect to the angle:

          $$E_{in_{\theta}}(R)=E_o\sin(\theta)$$
          $$E_{out_{\theta}}(R)=E_o\cos(\theta)$$

        \item We can then use Gauss's law:

          $$\frac{\sigma A}{\varepsilon_o}=A(E_{out_r}-E_{in_r})$$
          $$\sigma=\varepsilon_o(E_{out}(R,\theta)-E_{in}(R,\theta))$$
          $$\sigma=3\varepsilon_o(E_o\cos(\theta))$$

        \item This implies there is a positive charge at the north pole, a negative charge at the south pole, and no charge at the equator

        \item To find the total charge we can do the following:

          $$Q=\int_0^{\pi}R^2\sin(\theta)\,d\theta(3\varepsilon_oE_o\cos(\theta))=0$$

      \end{itemize}

    \item Poles:

      \begin{itemize}

        \item We can determine that:

          $$\left\{\begin{array}{l l}\text{Monopole: } & V\propto\frac{1}{r}\\\text{Dipole: } & V\propto\frac{1}{r^2}\\\text{Quadrupole: } & V\propto\frac{1}{r^3}\\\text{Octopole: } & V\propto\frac{1}{r^4}\end{array}$$

          \item The arrangement increases in dimension each time we go up an order; that, is, the shape is a single point charge for monopole, two equal but opposite charges along a line for dipole, a square with aggregate charge of zero for quadrupole, and a cube with aggregate charge zero for octopole

      \end{itemize}

\end{itemize}

\end{document}

