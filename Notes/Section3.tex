%%%%%%%%%%%%%%%%%%%%%%%%%%%%%%%%%%%%%%%%%%%%%%%%%%%%%%%%%%%%%%%%%%%%%%%%%%%%%%%%%%%%%%%%%%%%%%%%%%%%%%%%%%%%%%%%%%%%%%%%%%%%%%%%%%%%%%%%%%%%%%%%%%%%%%%%%%%%%%%%%%%
% Written By Michael Brodskiy
% Class: Electricity & Magnetism
% Professor: D. Wood
%%%%%%%%%%%%%%%%%%%%%%%%%%%%%%%%%%%%%%%%%%%%%%%%%%%%%%%%%%%%%%%%%%%%%%%%%%%%%%%%%%%%%%%%%%%%%%%%%%%%%%%%%%%%%%%%%%%%%%%%%%%%%%%%%%%%%%%%%%%%%%%%%%%%%%%%%%%%%%%%%%%

\documentclass[12pt]{article} 
\usepackage{alphalph}
\usepackage[utf8]{inputenc}
\usepackage[russian,english]{babel}
\usepackage{titling}
\usepackage{amsmath}
\usepackage{graphicx}
\usepackage{enumitem}
\usepackage{amssymb}
\usepackage[super]{nth}
\usepackage{everysel}
\usepackage{ragged2e}
\usepackage{geometry}
\usepackage{multicol}
\usepackage{fancyhdr}
\usepackage{cancel}
\usepackage{siunitx}
\usepackage{physics}
\usepackage{tikz}
\usepackage{mathdots}
\usepackage{yhmath}
\usepackage{cancel}
\usepackage{color}
\usepackage{array}
\usepackage{multirow}
\usepackage{gensymb}
\usepackage{tabularx}
\usepackage{extarrows}
\usepackage{booktabs}
\usepackage{lastpage}
\usetikzlibrary{fadings}
\usetikzlibrary{patterns}
\usetikzlibrary{shadows.blur}
\usetikzlibrary{shapes}

\geometry{top=1.0in,bottom=1.0in,left=1.0in,right=1.0in}
\newcommand{\subtitle}[1]{%
  \posttitle{%
    \par\end{center}
    \begin{center}\large#1\end{center}
    \vskip0.5em}%

}
\usepackage{hyperref}
\hypersetup{
colorlinks=true,
linkcolor=blue,
filecolor=magenta,      
urlcolor=blue,
citecolor=blue,
}


\title{Potentials}
\date{\today}
\author{Michael Brodskiy\\ \small Professor: D. Wood}

\begin{document}

\maketitle

\begin{itemize}

    $$\oint\vec{E}\cdot d\vec{a}=\frac{q_{enc}}{\varepsilon_o}\Leftrightarrow \vec{\nabla}\cdot\vec{E}=\frac{\rho}{\varepsilon_o}$$

  \item This means that, in a region with no charge, $\vec{\nabla}\cdot\vec{E}=0$, which also means $\nabla^2V=0$ (the Laplacian)

  \item Rectangular Coordinates:

    $$\nabla^2V=0\quad\quad \frac{\partial^2V}{\partial x^2}+\frac{\partial^2V}{\partial y^2}+\frac{\partial^2V}{\partial z^2}=0$$

    $$V(x,y,z)=X(x)Y(y)Z(z)\Rightarrow \frac{\partial^2X}{\partial x^2}YZ+\frac{\partial^2Y}{\partial y^2}XZ+\frac{\partial^2Z}{\partial z^2}XY=0$$

  \item If we divide by the respective functions, we get:

    $$\frac{\partial^2}{\partial x^2}\frac{1}{X}+\frac{\partial^2}{\partial y^2}\frac{1}{Y}+\frac{\partial^2}{\partial z^2}\frac{1}{Z}=0$$
    $$\frac{d^2X}{dx^2}=c_1X(x)\Rightarrow c_1>0:$$
    $$X(x)=\left\{\begin{array}{l r} Ae^{\pm kx} & c_1=k^2\\A\sin(kx) & c_1=-k^2\\A\cos(kx) & c_1=-k^2\end{array}$$

      \begin{itemize}

        \item Repeating this for each variable, we find

          $$\frac{d^2Y}{dy^2}=c_2Y(y)\quad\quad\frac{d^2Z}{dz^2}=c_2Z(z)$$
          $$c_1+c_2+c_3=0$$

      \end{itemize}

    \item Fourier Inversion

      $$\int_0^a\sin\left( \frac{n\pi y}{a} \right)\sin\left( \frac{m\pi y}{a} \right)\,dy=\frac{a}{2}\delta_{nm}$$

      This is used to define the terms for Fourier analysis

    \item A semi-infinite square tube, with the four walls grounded:

      \begin{itemize}

        \item The boundary conditions are (with $V=0$) given by:

          $$\left\{\begin{array}{l l} x=0, & 0<y<b\\x=a, & 0<y<b\\ y=0, & 0<x<a\\y=b, & 0<x<a\end{array}$$

          \item This would mean:

            $$X=\sin\left( \frac{n\pi x}{a} \right)\quad\quad Y\propto\sin\left( \frac{m\pi y}{b} \right)\quad\quad Z=e^{-k_zz}$$
            $$V=\sum B_{n,m}e^{-k_zz}\sin\left( \frac{n\pi x}{a} \right)\sin\left( \frac{m\pi y}{b} \right)$$

          \item Because we know $c_1+c_2+c_3=0$, we know:

            $$-\left( \frac{n\pi}{a} \right)^2-\left( \frac{m\pi}{b} \right)+k_z^2=0$$
            $$k_z=\sqrt{\left( \frac{n\pi}{a} \right)^2+\left( \frac{m\pi}{b} \right)^2}$$

      \end{itemize}

\end{itemize}

\end{document}

