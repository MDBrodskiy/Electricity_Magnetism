  %%%%%%%%%%%%%%%%%%%%%%%%%%%%%%%%%%%%%%%%%%%%%%%%%%%%%%%%%%%%%%%%%%%%%%%%%%%%%%%%%%%%%%%%%%%%%%%%%%%%%%%%%%%%%%%%%%%%%%%%%%%%%%%%%%%%%%%%%%%%%%%%%%%%%%%%%%%%%%%%%%%
% Written By Michael Brodskiy
% Class: Electricity & Magnetism
% Professor: D. Wood
%%%%%%%%%%%%%%%%%%%%%%%%%%%%%%%%%%%%%%%%%%%%%%%%%%%%%%%%%%%%%%%%%%%%%%%%%%%%%%%%%%%%%%%%%%%%%%%%%%%%%%%%%%%%%%%%%%%%%%%%%%%%%%%%%%%%%%%%%%%%%%%%%%%%%%%%%%%%%%%%%%%

\include{Includes.tex}

\title{Potentials and Fields}
\date{\today}
\author{Michael Brodskiy\\ \small Professor: D. Wood}

\begin{document}

\maketitle

\begin{itemize}

  \item The energy density of a field may be written as:

    $$U=\frac{1}{2}\left( \varepsilon_o\vec{E}^2+\frac{1}{\mu_o}\vec{B}^2 \right)$$

  \item The Poynting Vector may be defined as:

    $$\vec{S}=\frac{1}{\mu_o}(\vec{E}\times\vec{B})$$

  \item For some volume $\tau$, we can write the total energy in it using:

    $$\mathcal{U}_{tot}=\int_V U\,d\tau$$

  \item The Poynting Vector represents the flow of energy out of the surface

    $$\vec{\nabla}\cdot\vec{S}=\frac{1}{\mu_o}\vec{\nabla}\cdot(\vec{E}\times\vec{B})=\frac{1}{\mu_o}\left( \vec{B}\cdot(\vec{\nabla}\times\vec{E})-\vec{E}(\vec{\nabla}\times\vec{B}) \right)$$
    $$\vec{\nabla}\cdot\vec{S}=-\frac{\partial}{\partial t}\left( \frac{\varepsilon_o\vec{E}^2}{2}+\frac{\vec{B}^2}{2\mu_o} \right)$$

    \begin{itemize}

      \item This can be simplified to attain:

        $$\vec{\nabla}\cdot\vec{S}+\frac{\partial U}{\partial t}=0$$

      \item A similar equation can be seen in conservation of charge:

        $$\vec{\nabla}\cdot\vec{J}+\frac{\partial \rho}{\partial t}=0$$

      \item Thus, we have obtained a conservation of energy formula

    \end{itemize}

\end{itemize}

\end{document}

