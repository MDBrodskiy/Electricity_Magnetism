  %%%%%%%%%%%%%%%%%%%%%%%%%%%%%%%%%%%%%%%%%%%%%%%%%%%%%%%%%%%%%%%%%%%%%%%%%%%%%%%%%%%%%%%%%%%%%%%%%%%%%%%%%%%%%%%%%%%%%%%%%%%%%%%%%%%%%%%%%%%%%%%%%%%%%%%%%%%%%%%%%%%
% Written By Michael Brodskiy
% Class: Electricity & Magnetism
% Professor: D. Wood
%%%%%%%%%%%%%%%%%%%%%%%%%%%%%%%%%%%%%%%%%%%%%%%%%%%%%%%%%%%%%%%%%%%%%%%%%%%%%%%%%%%%%%%%%%%%%%%%%%%%%%%%%%%%%%%%%%%%%%%%%%%%%%%%%%%%%%%%%%%%%%%%%%%%%%%%%%%%%%%%%%%

\include{Includes.tex}

\title{Potentials and Fields}
\date{\today}
\author{Michael Brodskiy\\ \small Professor: D. Wood}

\begin{document}

\maketitle

\begin{itemize}

  \item The energy density of a field may be written as:

    $$U=\frac{1}{2}\left( \varepsilon_o\vec{E}^2+\frac{1}{\mu_o}\vec{B}^2 \right)$$

  \item The Poynting Vector may be defined as:

    $$\vec{S}=\frac{1}{\mu_o}(\vec{E}\times\vec{B})$$

  \item For some volume $\tau$, we can write the total energy in it using:

    $$\mathcal{U}_{tot}=\int_V U\,d\tau$$

  \item The Poynting Vector represents the flow of energy out of the surface

    $$\vec{\nabla}\cdot\vec{S}=\frac{1}{\mu_o}\vec{\nabla}\cdot(\vec{E}\times\vec{B})=\frac{1}{\mu_o}\left( \vec{B}\cdot(\vec{\nabla}\times\vec{E})-\vec{E}(\vec{\nabla}\times\vec{B}) \right)$$
    $$\vec{\nabla}\cdot\vec{S}=-\frac{\partial}{\partial t}\left( \frac{\varepsilon_o\vec{E}^2}{2}+\frac{\vec{B}^2}{2\mu_o} \right)$$

    \begin{itemize}

      \item This can be simplified to attain:

        $$\vec{\nabla}\cdot\vec{S}+\frac{\partial U}{\partial t}=0$$

      \item A similar equation can be seen in conservation of charge:

        $$\vec{\nabla}\cdot\vec{J}+\frac{\partial \rho}{\partial t}=0$$

      \item Thus, we have obtained a conservation of energy formula

    \end{itemize}

  \item The Maxwell Stress Tensor ($\overleftrightarrow{T}$)

    $$d\vec{F}=\overleftrightarrow{T}\cdot d\vec{a}$$

    \begin{itemize}

      \item $T_{ij}$ is the force per area in the $j$-direction of an area oriented in the $i$-direction

        $$\oint_S\overleftrightarrow{T}\cdot d\vec{a}=\vec{F}_{ext}$$
        $$\oint_V(\vec{\nabla}\cdot\overleftrightarrow{T})\cdot d\tau=\vec{F}_{ext}$$

      \item $T_{ij}=\varepsilon_o\left( E_iE_j-\frac{1}{2}\delta_{ij}E^2 \right)+\frac{1}{\mu_o}\left( B_iB_j-\frac{1}{2}\delta_{ij}B^2 \right)$

        $$\vec{\nabla}\cdot\overleftrightarrow{T}=\varepsilon_o\left( (\vec{\nabla}\cdot\vec{E} )\vec{E}+(\vec{E}\cdot\vec{\nabla})\vec{E}-\frac{1}{2}\vec{\nabla}\vec{E}^2\right)+\frac{1}{\mu_o}\left( (\vec{\nabla}\cdot\vec{B} )\vec{B}+(\vec{B}\cdot\vec{\nabla})\vec{B}-\frac{1}{2}\vec{\nabla}\vec{B}^2\right)$$
        $$\vec{\nabla}\cdot\overleftrightarrow{T}=\varepsilon_o\left( (\vec{\nabla}\cdot\vec{E} )\vec{E}-\vec{E}\times(\vec{\nabla}\times\vec{E})\right)+\frac{1}{\mu_o}\left( (\vec{\nabla}\cdot\vec{B} )\vec{B}-\vec{B}\times(\vec{\nabla}\times\vec{B})\right)$$

      \item Here, we see all four of Maxwell's laws. This lets us rewrite:

        $$\vec{\nabla}\cdot\overleftrightarrow{T}=\underbrace{\rho\vec{E}+\mu_o\vec{J}\times\vec{B}}_{\text{Force per Volume, $\vec{f}$}}+\underbrace{\varepsilon_o\vec{E}\times\left( \frac{\partial \vec{B}}{\partial t} \right)+\varepsilon_o\vec{B}\times\left( \frac{\partial \vec{E}}{\partial t} \right)}_{\frac{\partial}{\partial t}(\varepsilon_o\vec{E}\times\vec{B})}$$

      \item We can ultimately simplify to get:

        $$\vec{\nabla}\cdot\overleftrightarrow{T}=\vec{f}-\frac{\partial}{\partial t}\mu_o\varepsilon_o\vec{S}$$

      \item We can rewrite this as:

        $$\frac{d\vec{p}}{dt}=F_{ext}-\frac{\partial}{\partial t}\int\varepsilon_o\mu_o\vec{S}\,d\tau$$

      \item The momentum density may be written as:

        $$\vec{g}=\frac{\vec{p}}{V}=\varepsilon_o\mu_o\vec{S}=\varepsilon_o\vec{E}\times\vec{B}$$


      \item This can be summed up as:

        $$-\vec{\nabla}\cdot\overleftrightarrow{T}+\frac{\partial \vec{g}}{\partial t}=0$$

      \item Which is the continuity and conservation of momentum

    \end{itemize}

  \item Radiation Pressure

    \begin{itemize}

      \item For Absorption:

        $$P=\vec{g}c\cdot\bold{\hat{n}}$$

      \item For Reflection:

        $$P=2\vec{g}c\cdot\bold{\hat{n}}$$

      \item The time-averaged pressure may be defined as:

        $$\langle P\rangle=\frac{1}{T}\int_0^TP\cdot dt$$

    \end{itemize}

  \item Angular Momentum

    \begin{itemize}

      \item The angular momentum may be defined as:

        $$\vec{L}=\vec{r}\times\vec{p}$$

      \item To find the angular momentum density, we may write:

        $$\vec{l}=\vec{p}\times\vec{g}$$

      \item The torque on the inner cylinder of a coaxial cable may be written as:

        $$N=R_1qE_{\phi}=\frac{q\varepsilon}{2\pi}=-\frac{q}{2\pi}\frac{d\phi_1}{dt}$$

      \item The change in angular momentum may be written as:

        $$\Delta L_1=\frac{qB_iR_1^2}{2}$$

      \item For the outer cylinder, the change would be the same, except with $R_1\to R_2$

        $$\Delta L_{tot}=\frac{qB_i}{2}(R_1^2-R_2^2)$$
        $$\vec{l}=\frac{qB_i r}{2\pi s\Delta z}\hat{\theta}$$

    \end{itemize}

\end{itemize}

\end{document}

