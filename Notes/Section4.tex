%%%%%%%%%%%%%%%%%%%%%%%%%%%%%%%%%%%%%%%%%%%%%%%%%%%%%%%%%%%%%%%%%%%%%%%%%%%%%%%%%%%%%%%%%%%%%%%%%%%%%%%%%%%%%%%%%%%%%%%%%%%%%%%%%%%%%%%%%%%%%%%%%%%%%%%%%%%%%%%%%%%
% Written By Michael Brodskiy
% Class: Electricity & Magnetism
% Professor: D. Wood
%%%%%%%%%%%%%%%%%%%%%%%%%%%%%%%%%%%%%%%%%%%%%%%%%%%%%%%%%%%%%%%%%%%%%%%%%%%%%%%%%%%%%%%%%%%%%%%%%%%%%%%%%%%%%%%%%%%%%%%%%%%%%%%%%%%%%%%%%%%%%%%%%%%%%%%%%%%%%%%%%%%

\include{Includes.tex}

\title{Electric Fields in Matter}
\date{\today}
\author{Michael Brodskiy\\ \small Professor: D. Wood}

\begin{document}

\maketitle

\begin{itemize}

  \item Hydrogen Atom

    \begin{itemize}

      \item The charge density for a hydrogen atom would be:

        $$\rho(r)=\frac{q}{4\pi a^3}e^{-\frac{2r}{a}}$$

        when $r<<a$:

        $$\rho(r)\approx\frac{q}{4\pi a^3}$$

      \item Using Gauss's law, we can find the electric field:

        $$\vec{E}(4\pi r^2)=\frac{\rho(\frac{4}{3}\pi r^3)}{\varepsilon_o}$$
        $$\vec{E}=\frac{\rho r}{3\varepsilon_o}$$

      \item From here, the force on the charges become:

        $$F=q\vec{E}=-\frac{q\rho r}{3\varepsilon_o}$$

    \end{itemize}

  \item Molecule of Water

    \begin{itemize}

      \item Since the hydrogens pull in different directions, we can describe the torque as:

        $$\vec{N}=\frac{d}{2}\sin(\theta)\vec{E}q+\frac{d}{2}\sin(\theta)\vec{E}q=d\sin(\theta)\vec{E}q$$
        $$\vec{N}=\vec{p}\times\vec{E}$$

      \item The net force can be described as:

        $$F=q\vec{E}\left(\vec{r}+\frac{\vec{d}}{2}\right)-q\vec{E}\left( \vec{r}-\frac{\vec{d}}{2} \right)$$
        $$F_x\approx q\left( E_x(\vec{r})+\frac{\vec{d}}{2}\vec{\nabla}E_x \right)-q\left( \vec{E}(r)-\frac{\vec{d}}{2}\vec{\nabla} E_x \right)$$
        $$F_x=q\vec{\nabla}E_x\cdot\vec{d}=\vec{p}\cdot\vec{\nabla}E_x$$
        $$\vec{F}=(\vec{p}\cdot\vec{\nabla})\vec{E}$$

    \end{itemize}

  \item Polarized material

    $$\vec{P}=\frac{\text{dipole moment}}{\text{volume}}$$

    \begin{itemize}

      \item Units:

        $$\frac{\si{\coulomb}}{\si{\meter\squared}}\quad\text{ like }\sigma$$

      \item Potential due to a neutral polarized object

        $$V(\vec{r})=\frac{1}{4\pi\varepsilon_o}\int\frac{\bold{\hat{R}}\cdot P(\vec{r}\prime)}{R^2}\,d\tau'$$
        $$V(\vec{r})=-\frac{1}{4\pi\varepsilon_o}\int\frac{1}{R'}\vec{\nabla}\cdotP(\vec{r}\prime)\,d\tau'$$

    \end{itemize}

  \item For a bound charge:

    $$\sigma_b=\vec{P}\cdot\bold{\hat{n}}\quad\quad\text{bound surface charge}$$
    $$\rho_b=-\vec{\nabla}\cdot\vec{P}\quad\quad\text{bound bulk charge}$$

  \item Note, in general:

    $$\vec{E}\neq -\vec{\nabla}V_D$$

    \begin{itemize}

      \item This is due to the fact that:

        $$\vec{\nabla}\times\vec{D}\text{ is not always 0}$$

    \end{itemize}

  \item Special Rule for a linear dielectric

    $$\vec{P}=\chi_e\varepsilon_o\vec{E}$$

    \begin{itemize}

      \item The ``chi-sub-e'' value is its electric susceptibility (which is the ability for something to become electrically polarized)

      \item The permittivity is equal to:

        $$\varepsilon=(1+\chi_e)\varepsilon_o=\varepsilon_r\varepsilon_o$$

        \begin{itemize}

          \item We can recognize $\varepsilon_r$ as the dielectric constant

          \item $\varepsilon_o$ is the permittivity of free space

          \item $\varepsilon_r=1$ in a vacuum

            $$\vec{E}=\frac{\vec{D}}{\varepsilon}$$

        \end{itemize}

    \end{itemize}

  \item Capacitor with Linear Dielectric

    $$C=C_{vac}\varepsilon_r$$

    \begin{itemize}

      \item The energy becomes:

        $$W=\frac{1}{2}CV^2=\frac{1}{2}\varepsilon_rC_{vac}V^2$$

      \item The stored energy would be:

        $$U=\frac{W}{Ad}=\frac{1}{2}\varepsilon_r\varepsilon_o\vec{E}^2=\frac{1}{2}\vec{D}\vec{E}$$

    \end{itemize}

\end{itemize}

\end{document}

