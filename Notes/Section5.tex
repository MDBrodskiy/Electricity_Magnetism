%%%%%%%%%%%%%%%%%%%%%%%%%%%%%%%%%%%%%%%%%%%%%%%%%%%%%%%%%%%%%%%%%%%%%%%%%%%%%%%%%%%%%%%%%%%%%%%%%%%%%%%%%%%%%%%%%%%%%%%%%%%%%%%%%%%%%%%%%%%%%%%%%%%%%%%%%%%%%%%%%%%
% Written By Michael Brodskiy
% Class: Electricity & Magnetism
% Professor: D. Wood
%%%%%%%%%%%%%%%%%%%%%%%%%%%%%%%%%%%%%%%%%%%%%%%%%%%%%%%%%%%%%%%%%%%%%%%%%%%%%%%%%%%%%%%%%%%%%%%%%%%%%%%%%%%%%%%%%%%%%%%%%%%%%%%%%%%%%%%%%%%%%%%%%%%%%%%%%%%%%%%%%%%

\include{Includes.tex}

\title{Magnetostatics}
\date{\today}
\author{Michael Brodskiy\\ \small Professor: D. Wood}

\begin{document}

\maketitle

\begin{itemize}

  \item Magnetostatics:

    $$\vec{F}_{lorentz}=q\vec{v}\times\vec{B}$$

    \begin{itemize}

      \item This is ``static'' in the sense of steady flow of magnetic field

      \item The units of $\vec{B}$ are Teslas [$\si{\tesla}$]

      \item In the simple case where $\vec{B}=B_o\bold{\hat{z}}$ and $v_z=0$, the force could be described as:

        $$|\vec{F}|=qvB$$

      \item With inward direction. We can then write:

        $$qvB=\frac{mv^2}{R}$$
        $$qB=\frac{mv}{R}$$
        $$qB=\frac{p}{R}$$
        $$R=\frac{p}{qB}$$

    \end{itemize}

\end{itemize}

\end{document}

