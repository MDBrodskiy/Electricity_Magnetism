%%%%%%%%%%%%%%%%%%%%%%%%%%%%%%%%%%%%%%%%%%%%%%%%%%%%%%%%%%%%%%%%%%%%%%%%%%%%%%%%%%%%%%%%%%%%%%%%%%%%%%%%%%%%%%%%%%%%%%%%%%%%%%%%%%%%%%%%%%%%%%%%%%%%%%%%%%%%%%%%%%%
% Written By Michael Brodskiy
% Class: Electricity & Magnetism
% Professor: D. Wood
%%%%%%%%%%%%%%%%%%%%%%%%%%%%%%%%%%%%%%%%%%%%%%%%%%%%%%%%%%%%%%%%%%%%%%%%%%%%%%%%%%%%%%%%%%%%%%%%%%%%%%%%%%%%%%%%%%%%%%%%%%%%%%%%%%%%%%%%%%%%%%%%%%%%%%%%%%%%%%%%%%%

\documentclass[12pt]{article} 
\usepackage{alphalph}
\usepackage[utf8]{inputenc}
\usepackage[russian,english]{babel}
\usepackage{titling}
\usepackage{amsmath}
\usepackage{graphicx}
\usepackage{enumitem}
\usepackage{amssymb}
\usepackage[super]{nth}
\usepackage{everysel}
\usepackage{ragged2e}
\usepackage{geometry}
\usepackage{multicol}
\usepackage{fancyhdr}
\usepackage{cancel}
\usepackage{siunitx}
\usepackage{physics}
\usepackage{tikz}
\usepackage{mathdots}
\usepackage{yhmath}
\usepackage{cancel}
\usepackage{color}
\usepackage{array}
\usepackage{multirow}
\usepackage{gensymb}
\usepackage{tabularx}
\usepackage{extarrows}
\usepackage{booktabs}
\usepackage{lastpage}
\usetikzlibrary{fadings}
\usetikzlibrary{patterns}
\usetikzlibrary{shadows.blur}
\usetikzlibrary{shapes}

\geometry{top=1.0in,bottom=1.0in,left=1.0in,right=1.0in}
\newcommand{\subtitle}[1]{%
  \posttitle{%
    \par\end{center}
    \begin{center}\large#1\end{center}
    \vskip0.5em}%

}
\usepackage{hyperref}
\hypersetup{
colorlinks=true,
linkcolor=blue,
filecolor=magenta,      
urlcolor=blue,
citecolor=blue,
}


\title{Magnetostatics}
\date{\today}
\author{Michael Brodskiy\\ \small Professor: D. Wood}

\begin{document}

\maketitle

\begin{itemize}

  \item Magnetostatics:

    $$\vec{F}_{lorentz}=q\vec{v}\times\vec{B}$$

    \begin{itemize}

      \item This is ``static'' in the sense of steady flow of magnetic field

      \item The units of $\vec{B}$ are Teslas [$\si{\tesla}$]

      \item In the simple case where $\vec{B}=B_o\bold{\hat{z}}$ and $v_z=0$, the force could be described as:

        $$|\vec{F}|=qvB$$

      \item With inward direction. We can then write:

        $$qvB=\frac{mv^2}{R}$$
        $$qB=\frac{mv}{R}$$
        $$qB=\frac{p}{R}$$
        $$R=\frac{p}{qB}$$

      \item The frequency of rotation may be written:

        $$\omega=\frac{qB}{m}$$

      \item Total force may be written as:

        $$\vec{F}=q\vec{E}+q\vec{v}\times\vec{B}=q(\vec{E}+\vec{v}\times\vec{B})$$

        \begin{itemize}

          \item Special Case: $\vec{F}=0$

            $$\vec{E}+\vec{v}\times\vec{B}=0$$

          \item Given $\vec{E}=E_o\bold{\hat{z}}$ and $\vec{B}=B_o\bold{\hat{x}}$, we would get:

            $$E_o\bold{\hat{z}}+v_zB_o\bold{\hat{y}}-v_yB_o\bold{\hat{z}}=0$$

          \item From this, we get $v_z=0$ and $v_y=\frac{E_o}{B_o}$

          \item This can be used to construct a velocity selector

          \item In the event that $\vec{E}\parallel\vec{B}$, an expand or contracting helix about the fields as a pole would be constructed

        \end{itemize}

      \item We now go back to a general case: $\vec{E}\perp\vec{B}$, $\vec{F}_{net}\neq0$, $\vec{E}=E\bold{\hat{z}}$, and $\vec{B}=B\bold{\hat{x}}$

        $$\frac{1}{m}\frac{d\vec{v}}{dt}=q(\vec{E}+\vec{v}\times\vec{B})$$

        \begin{itemize}

          \item The solutions for this would look as follows:

            $$y(t)=c_1\cos(\omega t)+c_2\sin(\omega t)+\frac{E}{B}t+c_3$$
            $$z(t)=c_2\cos(\omega t)+c_1\sin(\omega t)+c_4$$
            $$\omega=\frac{qB}{m}$$

          \item This shape is known as a cycloid

          \item Special case: starting from rest at (0,0); this would give us:

            $$c_1=0\quad\quad c_2=-\frac{E}{\omega B}\quad\quad c_3=-c_1\quad\quad c_4=-c_2$$
            $$y(t)=-\frac{E}{\omega B}\sin(\omega t)+\frac{E}{B}t=R(\omega t-\sin(\omega t))$$
            $$z(t)=R(1-\cos(\omega t))$$
            $$\vec{v}_{cent}=\left( \frac{\vec{E}\times\vec{B}}{B^2} \right)$$

          \item Work:

            $$dW=\vec{F}\cdot d\vec{l}=(q\vec{v}\times\vec{B})\cdot \vec{v}\,dt=q\,dt(\vec{v}\times\vec{B})\cdot\vec{v}=0$$

          \item Critical note: magnetic fields \underline{do no work}

          \item Magnetic fields may induce electric fields to do work, but \underline{do not do work themselves}

        \end{itemize}

    \end{itemize}

  \item Continuous Systems

    \begin{itemize}

      \item Given a wire in space, with a bit of charge, $dq$, moving with velocity $\vec{v}$ shaped in rectangle, and placed in a magnetic field $\vec{B}=\frac{A}{z}\bold{\hat{x}}$:

        $$dq\,\vec{v}=I\,d\vec{l}$$
        $$d\vec{F}=I\,d\vec{l}\times\vec{B}$$
        $$\vec{F}_{mag}=\int I\,d\vec{l}\times\vec{B}$$

      \item Solving would give us:

        $$F_{tot}=-IL\left( \frac{A}{a} \right)\bold{\hat{z}}+\bold{\hat{y}}\int_a^b\frac{A}{z}\,dz+\bold{\hat{z}}IL\left( \frac{A}{b} \right)-\bold{\hat{y}}I\int_a^b\frac{A}{z}\,dz$$
        $$=-\bold{\hat{z}}ILA\left( \frac{1}{a}-\frac{1}{b} \right)$$

      \item There are several types of densities:

        $$\text{Linear:}\quad I\,d\vec{l}$$
        $$\text{Surface:}\quad \vec{K}\,da$$
        $$\text{Bulk:}\quad \vec{J}\,d\tau$$

    \end{itemize}

  \item The following is an important continuity equation:

    $$\vec{\nabla}\cdot\vec{J}+\frac{\partial\rho}{\partial t}=0$$

    \begin{itemize}

      \item This is another form of charge conservation

    \end{itemize}

  \item The Biot-Savart law is akin to Coulomb's law, but for magnetism:

    $$d\vec{B}=\frac{\mu_oI d\vec{l}\times \bold{\hat{R}}}{4\pi R^2}$$

    \begin{itemize}

      \item Not the expression can be modified:

        $$I\,d\vec{l}\Longleftrightarrow\vec{K}\,da\Longleftrightarrow\vec{J}\,d\tau$$

      \item The magnetic permeability of free space is:

        $$\mu_o=4\pi\cdot10^{-7}\left[ \frac{\si{\newton}}{\si{\ampere\squared}} \right]$$

        \begin{itemize}

          \item This term ``defines the amp'', and the current then ``defines the coulomb''

          \item Also note: the magnetic field is defined in Newtons per amp-meter

        \end{itemize}

    \end{itemize}

  \item Amp\`ere's Law

    $$\vec{\nabla}\times\vec{B}=\mu_oJ\Longleftrightarrow\vec{B}=\frac{\mu_o}{4\pi}\int\frac{\vec{J}\times\bold{\hat{R}}\,d\tau}{R^2}$$

    \begin{itemize}

      \item Notice, we can then use Stokes' Theorem to write:

        $$\oint\vec{B}\,dl=\int_V (\vec{\nabla}\times\vec{B})\,d\vec{a}=\int\mu_oJ^2\cdot\,d\vec{a}=\mu_oI_{enc}$$

    \end{itemize}

  \item Vector Potential

    \begin{itemize}

      \item We already know:

        $$\vec{E}=-\vec{\nabla}V$$

      \item We can derive a magnetic counterpart as:

        $$\vec{B}=\vec{\nabla}\times\vec{A}$$

      \item By definition, we know:

        $$\vec{\nabla}\times(\vec{A}+\vec{\nabla}t)=\vec{\nabla}\times\vec{A}+\underbrace{\vec{\nabla}\times\vec{\nabla}t}_{0}$$

        \begin{itemize}

          \item This is known as ``choice of gauge'' — we can expand upon this using:

        \end{itemize}

        $$\vec{\nabla}\cdot\vec{A}=\vec{\nabla}\cdot(\vec{A}+\vec{\nabla}t)=\vec{\nabla}\cdot\vec{A}+\nabla^2t$$
        $$\vec{\nabla}\cdot\vec{A}=0\to \nabla^2t=-\vec{\nabla}\cdot\vec{A}\prime$$

        \begin{itemize}

          \item This is known as the ``Coulomb gauge''

        \end{itemize}

      \item If we assume $\vec{\nabla}=0$ and use Amp\`ere's Law, we get:

        $$\vec{\nabla}\times\vec{B}=\mu_o\vec{J}$$
        $$\vec{\nabla}\times(\vec{\nabla}\times\vec{A})=\mu_o\vec{J}$$

        \begin{itemize}

          \item This then becomes

        \end{itemize}

        $$\vec{\nabla}\underbrace{(\vec{\nabla}\cdot\vec{A})}_{0}-\nabla^2\vec{A}=\mu_o\vec{J}$$
        $$\nabla^2\vec{A}=-\mu_o\vec{J}$$

      \item From the definition in electrostatics, we can obtain a definition for magnetostatics:

        $$V(\vec{r})=\frac{1}{4\pi\varepsilon_o}\int\frac{\rho(\vec{r}\prime)}{R}\,d\tau'$$
        $$\vec{A}(\vec{r})=\frac{\mu_o}{4\pi}\int\frac{\vec{J}(\vec{r}\prime)}{R}\,d\tau'$$

    \end{itemize}

\end{itemize}

\end{document}

