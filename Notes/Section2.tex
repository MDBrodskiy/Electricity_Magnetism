%%%%%%%%%%%%%%%%%%%%%%%%%%%%%%%%%%%%%%%%%%%%%%%%%%%%%%%%%%%%%%%%%%%%%%%%%%%%%%%%%%%%%%%%%%%%%%%%%%%%%%%%%%%%%%%%%%%%%%%%%%%%%%%%%%%%%%%%%%%%%%%%%%%%%%%%%%%%%%%%%%%
% Written By Michael Brodskiy
% Class: Electricity & Magnetism
% Professor: D. Wood
%%%%%%%%%%%%%%%%%%%%%%%%%%%%%%%%%%%%%%%%%%%%%%%%%%%%%%%%%%%%%%%%%%%%%%%%%%%%%%%%%%%%%%%%%%%%%%%%%%%%%%%%%%%%%%%%%%%%%%%%%%%%%%%%%%%%%%%%%%%%%%%%%%%%%%%%%%%%%%%%%%%

\documentclass[12pt]{article} 
\usepackage{alphalph}
\usepackage[utf8]{inputenc}
\usepackage[russian,english]{babel}
\usepackage{titling}
\usepackage{amsmath}
\usepackage{graphicx}
\usepackage{enumitem}
\usepackage{amssymb}
\usepackage[super]{nth}
\usepackage{everysel}
\usepackage{ragged2e}
\usepackage{geometry}
\usepackage{multicol}
\usepackage{fancyhdr}
\usepackage{cancel}
\usepackage{siunitx}
\usepackage{physics}
\usepackage{tikz}
\usepackage{mathdots}
\usepackage{yhmath}
\usepackage{cancel}
\usepackage{color}
\usepackage{array}
\usepackage{multirow}
\usepackage{gensymb}
\usepackage{tabularx}
\usepackage{extarrows}
\usepackage{booktabs}
\usepackage{lastpage}
\usetikzlibrary{fadings}
\usetikzlibrary{patterns}
\usetikzlibrary{shadows.blur}
\usetikzlibrary{shapes}

\geometry{top=1.0in,bottom=1.0in,left=1.0in,right=1.0in}
\newcommand{\subtitle}[1]{%
  \posttitle{%
    \par\end{center}
    \begin{center}\large#1\end{center}
    \vskip0.5em}%

}
\usepackage{hyperref}
\hypersetup{
colorlinks=true,
linkcolor=blue,
filecolor=magenta,      
urlcolor=blue,
citecolor=blue,
}


\title{Electrostatics}
\date{\today}
\author{Michael Brodskiy\\ \small Professor: D. Wood}

\begin{document}

\maketitle

\begin{itemize}

  \item In this section, we focus on electrostatics

    \begin{itemize}

      \item Not doing (for now):

        \begin{itemize}

          \item Magnetic field

          \item Forces on moving charges

          \item Finite of propagation

        \end{itemize}

    \end{itemize}

  \item Coulomb's Law

    \begin{itemize}

      \item Given a source charge, $q$, and a test charge, $Q$, with $\vec{R}$ as the difference between their positions ($\vec{r}-\vec{r}'$), we can generate Coulomb's Law:

        $$F=\frac{qQ\bold{\hat{R}}}{4\pi\varepsilon_oR^2}$$

        \begin{itemize}

          \item $\varepsilon_o$ is known as the permittivity of free space

          \item $\varepsilon_o=8.85\cdot10^{-12}\left[\frac{\si{\coulomb}}{\si{\newton\meter\squared}}\right]$

        \end{itemize}

      \item A Coulomb is defined as an Amp\`ere per second

    \end{itemize}

  \item Superposition

    \begin{itemize}

      \item A force per charge  ($q$) can be calculated and then summed to find the total force on a test charge ($Q$)

        $$\vec{F}=\sum_n\frac{Q}{4\pi\varepsilon_o}\frac{q_n\bold{\hat{R}}_n}{R_n^2}$$

        $$\vec{F}=Q\vec{E}\Rightarrow \vec{E}=\frac{1}{4\pi\varepsilon_o}\sum_n\frac{q_n\bold{\hat{R}}_n}{R_n^2}$$

    \end{itemize}

  \item Continuous

    $$q_n\rightarrow dq=\rho(\vec{r\prime})\,d\tau$$
    $$\vec{E}=\frac{1}{4\pi\varepsilon_o}\int\frac{1}{R^2}\bold{\hat{R}}\,dq=\frac{1}{4\pi\varepsilon_o}\int \frac{\rho(\vec{r\prime})\,d\tau'}{R^2}\bold{\hat{R}}$$

  \item For various shapes:

    \begin{itemize}

      \item Volume: $dq=\rho\,d\tau$

      \item Line: $dq=\lambda\,dl$

      \item Surface: $dq=\sigm\,da$

    \end{itemize}

  \item Electric Potential — $V$ (volts)

      $$\vec{E}=-\vec{\nabla}V \Longleftrightarrow V_b-V_a=-\int_a^b \vec{E}\cdot d\vec{l}$$

    \begin{itemize}

      \item Note: $V$ is a scalar function

      \item For a charge $q$ and some reference radius, $r$:

        $$V(r)-V_{ref}=-\int_{r_{ref}}^r \vec{E}\cdot d\vec{r\prime}$$

      \item This yields

        $$V(r)=\frac{q}{4\pi\varepsilon_o r}$$

      \item With multiple charges:

        $$V(r)=\frac{1}{4\pi\varepsilon_o}\sum_n\frac{q_n}{R_n}$$

    \end{itemize}

  \item Taylor Series Expansions

    \begin{itemize}

      \item We can write the expression $\frac{1}{x-L}$, where $\frac{L}{x}<<1$, as:

        $$\frac{1}{x-L}=\frac{1}{x}\left( 1-\frac{L}{x} \right)^{-1}\approx\frac{1}{x}+\frac{L}{x^2}+\ldots$$

        This means:

        $$\frac{1}{x-L}-\frac{1}{x}\approx\frac{L}{x^2}$$

    \end{itemize}

  \item Coulomb's Law with Gauss's Theorem

    \begin{itemize}

      \item Develops into Gauss's Law


        $$\oint_S\vec{E}\cdot d\vec{a}=\int_V\vec{\nabla}\cdot\vec{E}\cdot d\tau$$
        $$\oint\vec{E}\cdot d\vec{a}=\sum\underbrace{\int_V \frac{q_n}{\varepsilon_o}\delta^3(R_n)}_{\text{enc charge over $\varepsilon_o$}}$$
        $$\oint\vec{E}\cdot d\vec{a}=\frac{q_{enc}}{\varepsilon_o}$$

      \item Gauss's Law:

        \begin{itemize}

          \item Exploit symmetry

          \item Large or small distance to approximate symmetry
            
        \end{itemize}

    \end{itemize}

\end{itemize}

\end{document}

