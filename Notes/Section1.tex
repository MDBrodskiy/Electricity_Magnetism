%%%%%%%%%%%%%%%%%%%%%%%%%%%%%%%%%%%%%%%%%%%%%%%%%%%%%%%%%%%%%%%%%%%%%%%%%%%%%%%%%%%%%%%%%%%%%%%%%%%%%%%%%%%%%%%%%%%%%%%%%%%%%%%%%%%%%%%%%%%%%%%%%%%%%%%%%%%%%%%%%%%
% Written By Michael Brodskiy
% Class: Electricity & Magnetism
% Professor: D. Wood
%%%%%%%%%%%%%%%%%%%%%%%%%%%%%%%%%%%%%%%%%%%%%%%%%%%%%%%%%%%%%%%%%%%%%%%%%%%%%%%%%%%%%%%%%%%%%%%%%%%%%%%%%%%%%%%%%%%%%%%%%%%%%%%%%%%%%%%%%%%%%%%%%%%%%%%%%%%%%%%%%%%

\documentclass[12pt]{article} 
\usepackage{alphalph}
\usepackage[utf8]{inputenc}
\usepackage[russian,english]{babel}
\usepackage{titling}
\usepackage{amsmath}
\usepackage{graphicx}
\usepackage{enumitem}
\usepackage{amssymb}
\usepackage[super]{nth}
\usepackage{everysel}
\usepackage{ragged2e}
\usepackage{geometry}
\usepackage{multicol}
\usepackage{fancyhdr}
\usepackage{cancel}
\usepackage{siunitx}
\usepackage{physics}
\usepackage{tikz}
\usepackage{mathdots}
\usepackage{yhmath}
\usepackage{cancel}
\usepackage{color}
\usepackage{array}
\usepackage{multirow}
\usepackage{gensymb}
\usepackage{tabularx}
\usepackage{extarrows}
\usepackage{booktabs}
\usepackage{lastpage}
\usetikzlibrary{fadings}
\usetikzlibrary{patterns}
\usetikzlibrary{shadows.blur}
\usetikzlibrary{shapes}

\geometry{top=1.0in,bottom=1.0in,left=1.0in,right=1.0in}
\newcommand{\subtitle}[1]{%
  \posttitle{%
    \par\end{center}
    \begin{center}\large#1\end{center}
    \vskip0.5em}%

}
\usepackage{hyperref}
\hypersetup{
colorlinks=true,
linkcolor=blue,
filecolor=magenta,      
urlcolor=blue,
citecolor=blue,
}


\title{Vector Calculus}
\date{\today}
\author{Michael Brodskiy\\ \small Professor: D. Wood}

\begin{document}

\maketitle

\begin{itemize}

  \item A vector is defined by:

    \begin{itemize}

      \item Transformation under rotation

        $$\left( \begin{array}{c} A_x'\\A_y'\end{array} \right)=\left( \begin{array}{c c} \mathbb{R}_{xx} & \mathbb{R}_{yy}\\ \mathbb{R}_{yx} & \mathbb{R}_{yy}\end{array} \right)\left( \begin{array}{c} A_x\\ A_y\end{array} \right)$$
      $$\mathbb{R}=\left( \begin{array}{c c}\cos(\delta\phi) & -\sin(\delta\phi)\\ \sin(\delta\phi) & \cos(\delta\phi) \end{array} \right)$$

      \item Examples include electric fields, magnetic fields, momentum, displacement, etc.

    \end{itemize}

  \item Scalars

    \begin{itemize}

      \item Invariant under rotation

      \item Examples include charge, mass, electric potential, energy, etc.

    \end{itemize}

  \item Tensors (rank 2)

    \begin{itemize}

      \item $\mathbb{R}$ above is an example

    \end{itemize}

  \item Differential Operators

    \begin{itemize}

      \item Gradient $\longrightarrow \vec{\nabla}=\bold{hat{x}}\frac{\partial}{\partial x}+\bold{\hat{y}}\frac{\partial}{\partial y} + \bold{\hat{z}}\frac{\partial}{\partial z}$

        \begin{itemize}

          \item Must operate on something to be useful

          \item Ex. $\vec{E}=-\vec{\nabla}V(x,y,z)$

        \end{itemize}

    \end{itemize}

  \item Maxwell's Equations in a Vacuum (in SI units)

    $$\vec{\nabla}\cdot\vec{E}=\frac{1}{\epsilon_o}\rho$$
    $$\vec{\nabla}\times\vec{E}=-\frac{\partial\vec{B}}{\partial t}$$
    $$\vec{\nabla}\cdot\vec{B}=0$$
    $$\vec{\nabla}\times\vec{B}=\mu_o\vec{J}+\mu_o\epsilon_o\frac{\partial\vec{E}}{\partial t}$$

    \begin{itemize}

      \item SI units for E\&M: Coulomb, Volt, Tesla, Ampere

    \end{itemize}

  \item Force between two objects

    \begin{itemize}

      \item In SI:

        $$\vec{F}_{12}=\frac{q_1q_2(\widehat{r_1-r_2})}{4\pi\epsilon_o r_{12}^2}$$

      \item In CGS:

        $$F=\frac{q_1q_2}{r^2}$$

    \end{itemize}

  \item Cross Products

    $$\vec{A}\times\vec{B}=\left|\begin{matrix}\hat{i} & \hat{j} & \hat{k}\\ A_x & A_y & A_z\\ B_x & B_y & B_z\end{matrix}\right|$$

    \begin{itemize}

      \item Not cumulative:

    $$\vec{A}\times\vec{B}=-\vec{B}\times\vec{A}$$

      \item Distributive:

        $$\vec{A}\times(\vec{B}+\vec{C})=\vec{A}\times\vec{B}+\vec{A}\times\vec{C}$$

      \item Not associative:

        $$(\vec{A}\times\vec{B})\times\vec{C}\neq\vec{A}\times(\vec{B}\times\vec{C})$$

    \end{itemize}

  \item Unit Vectors:

    $$\hat{v}=\frac{\vec{v}}{|\vec{v}|}$$

  \item Gradient of a scalar field

    \begin{itemize}

      \item If $T$ is a scalar field, then:

        $$\vec{\nabla}T=\frac{\partial T}{\partial x}\bold{\hat{x}}+\frac{\partial T}{\partial y}\bold{\hat{y}}+\frac{\partial T}{\partial z}\bold{\hat{z}}$$

      \item Ex. $T=y^2z$

        $$\vec{\nabla}T=0\hat{x}+(2yz)\hat{y}+(y^2)\hat{z}$$

      \item Ex. $T=r^3=(x^2+y^2+z^2)^{\frac{3}{2}}$

        $$\vec{\nabla}T=\left(\frac{3}{2}(x^2+y^2+z^2)^{\frac{1}{2}}(2x)\right)\hat{x}+\left(\frac{3}{2}(x^2+y^2+z^2)^{\frac{1}{2}}(2y)\right)\hat{y}$$
        $$+\left(\frac{3}{2}(x^2+y^2+z^2)^{\frac{1}{2}}(2z)\right)\hat{z}$$
        $$\vec{\nabla}T=3r(x\hat{x}+y\hat{y}+z\hat{z})=3r\vec{r}=3r^2\hat{r}$$
        $$\vec{\nabla}(r^3)=3r^2(\vec{\nabla}r)$$

        Thus, we see:

        $$\vec{\nabla}r=\hat{r}$$

        Think in terms of dimensionality.

    \end{itemize}

  \item Product Rule

    \begin{itemize}

      \item In One Dimension:

        $$\frac{d}{dx}(fg)=\frac{df}{dx}g+f\frac{dg}{dx}$$

      \item Three Dimensions:

        $$\vec{\nabla}(fg)=(\vec{\nabla}f)g+f(\vec{\nabla}g)=g\vec{\nabla}f+f\vec{\nabla}g$$

        where $f,g$ are scalar functions of $x,y,z$

      \item Where $a$ is constant:

        $$\vec{\nabla}(af)=a\vec{\nabla}(f)$$

    \end{itemize}

  \item Chain Rule

    \begin{itemize}

      \item In One Dimension:

        $$\frac{d}{dx}(f(g(x)))=f'(g(x))\frac{dg}{dx}=\frac{\partial f}{\partial g}\frac{\partial g}{\partial x}$$

      \item Three Dimensions:

        $$\vec{\nabla}(f(g(x,y,z)))=\frac{\partial f}{\partial g}\vec{\nabla}g$$

      \item Example:

        $$f(g)=g^3,\,g=r=\sqrt{x^2+y^2+z^2}$$
        $$\frac{\partial f}{\partial g}\vec{\nabla}g=3g^2\vec{\nabla}(r)=3r^2\hat{r}$$

    \end{itemize}

  \item Divergence (where $\vec{v}$ is a vector function)

      $$\vec{\nabla}\cdot\vec{v}=\frac{\partial v_x}{\partial x}+\frac{\partial v_y}{\partial y}+\frac{\partial v_z}{\partial z}$$

    \begin{itemize}

      \item Example of positive divergence (where $\vec{v}=x\bold{\hat{x}}$). Looking at the graph of the vector field and taking a sample volume, there is more going ``out'' than ``in,'' which indicates that the divergence is greater than 0

        $$\vec{\nabla}\cdot\vec{v}=1 + 0 + 0=1$$

      \item Zero divergence would mean the same quantity ``out'' as ``in,'' like when $\vec{v}$ is a constant in any direction

      \item Negative divergence 

        $$\vec{v}=\frac{\bold{\hat{r}}}{r^3}$$
        $$\vec{\nabla}\cdot\frac{\bold{\hat{r}}}{r^3}=\vec{\nabla}\cdot\frac{\vec{r}}{r^4}=(\vec{\nabla}\cdot\vec{r})\frac{1}{r^4}+\vec{r}\left( \vec{\nabla}\frac{1}{r^4} \right)=\frac{3}{r^4}+\vec{r}\left( -\frac{4\bold{\hat{r}}}{r^5} \right)=-\frac{1}{r^4}\footnote{Keep in mind, $r\bold{\hat{r}}=\vec{r}$, and $\vec{r}=x\bold{\hat{x}}+y\bold{\hat{y}}+z\bold{\hat{z}}$}$$

    \end{itemize}
    
  \item Curl

    $$\vec{\nabla}\times\vec{v}=\text{curl}(\vec{v})=\left|\begin{matrix} \bold{\hat{x}} & \bold{\hat{y}} & \bold{\hat{z}}\\ \frac{\partial}{\partial x} & \frac{\partial}{\partial y} & \frac{\partial}{\partial z} \\ v_x & v_y & v_z\end{matrix}\right|$$

    \begin{itemize}

      \item Product rule for curl:

        \begin{itemize}

          \item Scalar times vector:

            $$\vec{\nabla}\times(f(x,y,z)\vec{A}(x,y,z))=\vec{\nabla}f\times A+f(\vec{\nabla}\times\vec{A})$$

          \item Vector times vector:

            $$\vec{\nabla}\times(\vec{A}\times\vec{B})=(\vec{B}\cdot\vec{\nabla})\vec{A}-(\vec{A}\cdot\vec{\nabla})\vec{B}+\vec{A}(\vec{\nabla}\cdot\vec{B})-\vec{B}(\vec{\nabla}\cdot\vec{A})$$

        \end{itemize}

    \end{itemize}

  \item The Laplacian

    \begin{itemize}

      \item $\nabla^2 T=\vec{\nabla}\cdot \vec{\nabla}T=\displaystyle \frac{\partial^2T}{\partial x^2}+\frac{\partial^2T}{\partial y^2}+\frac{\partial^2T}{\partial z^2}$

      \item $\nabla^2 \vec{v}=\nabla^2v_x\bold{\hat{x}}+\nabla^2v_y\bold{\hat{y}}+\nabla^2v_z\bold{\hat{z}}$

    \end{itemize}

  \item The Fundamental Theorem of Calculus

  \item In one dimension

    $$\int_a^b f(x)\,dx=f(b)-f(a)$$

  \item In three dimensions:

    $$\int_{\vec{a}}^{\vec{b}}\vec{\nabla}T\,\cdot d\vec{l}=T(\vec{b})-T(\vec{a})$$

    \begin{itemize}

      \item Where $d\vec{l}=dx\bold{\hat{x}}+dy\bold{\hat{y}}+dz\bold{\hat{z}}$

      \item This means that the above formula is independent of path; that is:

        $$\oint\vec{\nabla}T\,\cdot d\vec{l}=\int_{\vec{a}}^{\vec{b}}\vec{\nabla} T\,\cdot d\vec{l}$$

    \end{itemize}

  \item Fundamental Theorem for Divergence

    $$\underbrace{\int\vec{\nabla}\cdot\vec{N}d\tau}_{\text{volume}}=\underbrace{\oint\vec{v}\cdot d\vec{a}}_{\text{surface}}$$

    \begin{itemize}

      \item $d\tau$ refers to the differential volume element (that is, $dx\,dy\,dz$, $r^2\sin(\theta)\,dr\,d\theta\,d\phi$, etc.)

    \end{itemize}

  \item Fundamental Theorem for Curl (Stoke's Theorem)

    $$\underbrace{\oint\vec{v}\cdot d\vec{l}}_{\text{boundary}}=\underbrace{\int(\vec{\nabla}\times\vec{v})\cdot d\vec{a}}_{\text{surface}}$$

    \begin{center}
      \begin{tabular}[h]{|c|c|}
        \hline
        Object & Boundary\\
        \hline
        Line & End Points\\
        \hline
        Open Surface & Perimeter\\
        \hline
        Volume & Surface\\
        \hline
      \end{tabular}
    \end{center}

  \item Spherical Coordinates

    \begin{itemize}

      \item In spherical coordinates, we obtain the following transformation:

        $$\left\{\begin{array}{l}x=r\sin(\theta)\cos(\phi)\\y=r\sin(\theta)\sin(\phi)\\z=r\cos(\theta)\end{array}$$

        \item Whereas $d\vec{l}=dx\bold{\hat{x}}+dy\bold{\hat{y}}+dz\bold{\hat{z}}$ in rectangular, in spherical coordinates, $d\vec{l}=dr\bold{\hat{r}}+r\,d\theta\bold{\hat{\theta}}+r\sin(\theta)\,d\phi\bold{\hat{\phi}}$

        \item This makes $d\tau=r^2\sin(\theta)\,dr\,d\theta\,d\phi$

        \item The gradient becomes:

          $$\vec{\nabla}=\bold{\hat{r}}\frac{\partial}{\partial r}+\frac{\bold{\hat{\theta}}}{r}\frac{\partial}{\partial \theta}+\frac{\bold{\hat{\phi}}}{r\sin(\theta)}\frac{\partial}{\partial \phi}$$

        \item The divergence becomes:

          $$\vec{\nabla}\cdot\vec{v}=\frac{1}{r^2}\frac{\partial}{\partial r}(r^2v_r)+\frac{1}{r\sin(\theta)}\frac{\partial}{\partial \theta}(\sin(\theta)v_{\theta})+\frac{1}{r\sin(\theta)}\frac{\partial v_{\phi}}{\partial \phi}$$

        \item The Laplacian becomes:

          $$\nabla^2 T=\frac{1}{r^2}\frac{\partial}{\partial r}\left( r^2\frac{\partial T}{\partial r} \right)+\frac{1}{r^2\sin(\theta)}\frac{\partial}{\partial \theta}\left( \sin(\theta)\frac{\partial T}{\partial \theta} \right)+\frac{1}{r^2\sin^2(\theta)}\frac{\partial^2T}{\partial\phi^2}$$

    \end{itemize}

  \item Delta Functions

    \begin{itemize}

      \item In one dimension:

        $$\int_b^c \delta(x-a)=\left\{\begin{array}{l}1,\quad b<a<c\\0,\quad\text{otherwise}\end{array}$$

      \item In three dimensions:

        $$\int_{\text{vol}} \delta^3(\vec{r}-\vec{a})\,dx\,dy\,dz=\left\{\begin{array}{l}1,\quad\vec{a}\text{ in volume}\\0,\quad\text{otherwise}\end{array}$$

        \item Note:

          $$\vec{\nabla}\left( \frac{\bold{\hat{r}}}{r^2} \right)=0,\text{ for }r\neq0$$

    \end{itemize}

\end{itemize}

\end{document}

