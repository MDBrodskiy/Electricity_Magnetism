%%%%%%%%%%%%%%%%%%%%%%%%%%%%%%%%%%%%%%%%%%%%%%%%%%%%%%%%%%%%%%%%%%%%%%%%%%%%%%%%%%%%%%%%%%%%%%%%%%%%%%%%%%%%%%%%%%%%%%%%%%%%%%%%%%%%%%%%%%%%%%%%%%%%%%%%%%%%%%%%%%%
% Written By Michael Brodskiy
% Class: Electricity & Magnetism
% Professor: D. Wood
%%%%%%%%%%%%%%%%%%%%%%%%%%%%%%%%%%%%%%%%%%%%%%%%%%%%%%%%%%%%%%%%%%%%%%%%%%%%%%%%%%%%%%%%%%%%%%%%%%%%%%%%%%%%%%%%%%%%%%%%%%%%%%%%%%%%%%%%%%%%%%%%%%%%%%%%%%%%%%%%%%%

\include{Includes.tex}

\title{Vector Calculus}
\date{\today}
\author{Michael Brodskiy\\ \small Professor: D. Wood}

\begin{document}

\maketitle

\begin{itemize}

  \item A vector is defined by:

    \begin{itemize}

      \item Transformation under rotation

        $$\left( \begin{array}{c} A_x'\\A_y'\end{array} \right)=\left( \begin{array}{c c} \mathbb{R}_{xx} & \mathbb{R}_{yy}\\ \mathbb{R}_{yx} & \mathbb{R}_{yy}\end{array} \right)\left( \begin{array}{c} A_x\\ A_y\end{array} \right)$$
      $$\mathbb{R}=\left( \begin{array}{c c}\cos(\delta\phi) & -\sin(\delta\phi)\\ \sin(\delta\phi) & \cos(\delta\phi) \end{array} \right)$$

      \item Examples include electric fields, magnetic fields, momentum, displacement, etc.

    \end{itemize}

  \item Scalars

    \begin{itemize}

      \item Invariant under rotation

      \item Examples include charge, mass, electric potential, energy, etc.

    \end{itemize}

  \item Tensors (rank 2)

    \begin{itemize}

      \item $\mathbb{R}$ above is an example

    \end{itemize}

  \item Differential Operators

    \begin{itemize}

      \item Gradient $\longrightarrow \vec{\nabla}=\hat{x}\frac{\partial}{\partial x}+\hat{y}\frac{\partial}{\partial y} + \hat{z}\frac{\partial}{\partial z}$

        \begin{itemize}

          \item Must operate on something to be useful

          \item Ex. $\vec{E}=-\vec{\nabla}V(x,y,z)$

        \end{itemize}

    \end{itemize}

  \item Maxwell's Equations in a Vacuum (in SI units)

    $$\vec{\nabla}\cdot\vec{E}=\frac{1}{\epsilon_o}\rho$$
    $$\vec{\nabla}\times\vec{E}=-\frac{\partial\vec{B}}{\partial t}$$
    $$\vec{\nabla}\cdot\vec{B}=0$$
    $$\vec{\nabla}\times\vec{B}=\mu_o\vec{J}+\mu_o\epsilon_o\frac{\partial\vec{E}}{\partial t}$$

    \begin{itemize}

      \item SI units for E\&M: Coulomb, Volt, Tesla, Ampere

    \end{itemize}

  \item Force between two objects

    \begin{itemize}

      \item In SI:

        $$\vec{F}_{12}=\frac{q_1q_2(\widehat{r_1-r_2})}{4\pi\epsilon_o r_{12}^2}$$

      \item In CGS:

        $$F=\frac{q_1q_2}{r^2}$$

    \end{itemize}

  \item Cross Products

    $$\vec{A}\times\vec{B}=\left|\begin{matrix}\hat{i} & \hat{j} & \hat{k}\\ A_x & A_y & A_z\\ B_x & B_y & B_z\end{matrix}\right|$$

    \begin{itemize}

      \item Not cumulative:

    $$\vec{A}\times\vec{B}=-\vec{B}\times\vec{A}$$

      \item Distributive:

        $$\vec{A}\times(\vec{B}+\vec{C})=\vec{A}\times\vec{B}+\vec{A}\times\vec{C}$$

      \item Not associative:

        $$(\vec{A}\times\vec{B})\times\vec{C}\neq\vec{A}\times(\vec{B}\times\vec{C})$$

    \end{itemize}

  \item Unit Vectors:

    $$\hat{v}=\frac{\vec{v}}{|\vec{v}|}$$

  \item Gradient of a scalar field

    \begin{itemize}

      \item If $T$ is a scalar field, then:

        $$\vec{\nabla}T=\frac{\partial T}{\partial x}\hat{x}+\frac{\partial T}{\partial y}\hat{y}+\frac{\partial T}{\partial z}\hat{z}$$

      \item Ex. $T=y^2z$

        $$\vec{\nabla}T=0\hat{x}+(2yz)\hat{y}+(y^2)\hat{z}$$

      \item Ex. $T=r^3=(x^2+y^2+z^2)^{\frac{3}{2}}$

        $$\vec{\nabla}T=\left(\frac{3}{2}(x^2+y^2+z^2)^{\frac{1}{2}}(2x)\right)\hat{x}+\left(\frac{3}{2}(x^2+y^2+z^2)^{\frac{1}{2}}(2y)\right)\hat{y}$$
        $$+\left(\frac{3}{2}(x^2+y^2+z^2)^{\frac{1}{2}}(2z)\right)\hat{z}$$
        $$\vec{\nabla}T=3r(x\hat{x}+y\hat{y}+z\hat{z})=3r\vec{r}=3r^2\hat{r}$$
        $$\vec{\nabla}(r^3)=3r^2(\vec{\nabla}r)$$

        Thus, we see:

        $$\vec{\nabla}r=\hat{r}$$

        Think in terms of dimensionality.

    \end{itemize}

\end{itemize}

\end{document}

