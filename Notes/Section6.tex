%%%%%%%%%%%%%%%%%%%%%%%%%%%%%%%%%%%%%%%%%%%%%%%%%%%%%%%%%%%%%%%%%%%%%%%%%%%%%%%%%%%%%%%%%%%%%%%%%%%%%%%%%%%%%%%%%%%%%%%%%%%%%%%%%%%%%%%%%%%%%%%%%%%%%%%%%%%%%%%%%%%
% Written By Michael Brodskiy
% Class: Electricity & Magnetism
% Professor: D. Wood
%%%%%%%%%%%%%%%%%%%%%%%%%%%%%%%%%%%%%%%%%%%%%%%%%%%%%%%%%%%%%%%%%%%%%%%%%%%%%%%%%%%%%%%%%%%%%%%%%%%%%%%%%%%%%%%%%%%%%%%%%%%%%%%%%%%%%%%%%%%%%%%%%%%%%%%%%%%%%%%%%%%

\documentclass[12pt]{article} 
\usepackage{alphalph}
\usepackage[utf8]{inputenc}
\usepackage[russian,english]{babel}
\usepackage{titling}
\usepackage{amsmath}
\usepackage{graphicx}
\usepackage{enumitem}
\usepackage{amssymb}
\usepackage[super]{nth}
\usepackage{everysel}
\usepackage{ragged2e}
\usepackage{geometry}
\usepackage{multicol}
\usepackage{fancyhdr}
\usepackage{cancel}
\usepackage{siunitx}
\usepackage{physics}
\usepackage{tikz}
\usepackage{mathdots}
\usepackage{yhmath}
\usepackage{cancel}
\usepackage{color}
\usepackage{array}
\usepackage{multirow}
\usepackage{gensymb}
\usepackage{tabularx}
\usepackage{extarrows}
\usepackage{booktabs}
\usepackage{lastpage}
\usetikzlibrary{fadings}
\usetikzlibrary{patterns}
\usetikzlibrary{shadows.blur}
\usetikzlibrary{shapes}

\geometry{top=1.0in,bottom=1.0in,left=1.0in,right=1.0in}
\newcommand{\subtitle}[1]{%
  \posttitle{%
    \par\end{center}
    \begin{center}\large#1\end{center}
    \vskip0.5em}%

}
\usepackage{hyperref}
\hypersetup{
colorlinks=true,
linkcolor=blue,
filecolor=magenta,      
urlcolor=blue,
citecolor=blue,
}


\title{Magnetic Fields in Matter}
\date{\today}
\author{Michael Brodskiy\\ \small Professor: D. Wood}

\begin{document}

\maketitle

\begin{itemize}

  \item Torque on a Dipole

    $$\vec{N}=\vec{m}\times\vec{B}$$

    \begin{itemize}

      \item This is akin to the torque on an electric dipole:

        $$\vec{N}=\vec{p}\times\vec{E}$$

      \item We can calculate the torque to be:

        $$\vec{N}=m\vec{B}\sin(\theta)$$

      \item The energy can be defined as:

        $$U=-\vec{m}\cdot\vec{B}$$

      \item The bound bulk and surface currents can be defined as:

        $$\vec{J}_b=\vec{\nabla}\times\vec{M}\quad\text{ and }\quad\vec{K}_b=\vec{M}\times\bold{\hat{n}}$$

      \item For a uniform magnetized sphere:

        $$\vec{J}_b=0\quad\text{ and }\quad\vec{K}_b=M\sin(\theta)\hat{\phi}$$

      \item

        $$\vec{A}(r)=\frac{\mu_o}{4\pi}\int\frac{\vec{J}_b}{R}\,d\tau+\frac{\mu_o}{4\pi}\int\frac{\vec{K}_b}{R}\,da$$

    \end{itemize}

  \item Auxiliary Field ($\vec{H}$)

    $$\vec{H}=\frac{1}{\mu_o}\vec{B}-\vec{M}\Rightarrow \vec{B}=\mu(\vec{H}+\vec{M})$$

    \begin{itemize}

      \item From this formula, we can get:

        $$\vec{\nabla}\times\vec{H}=\frac{\vec{\nabla}\times\vec{B}}{\mu_o}-\vec{\nabla}\times\vec{M}$$

      \item Which can become:

        $$\vec{\nabla}\times\vec{H}=\vec{J}-\vec{J}_b=\vec{J}_F$$

    \end{itemize}

  \item Linear Materials

    $$\vec{M}=\chi_m\vec{H}$$

    \begin{itemize}

      \item This is not quite the same as the $\vec{E}$ case, where $\vec{p}=\varepsilon_o\chi_e\vec{E}$


        $$\vec{B}=\mu_o(\vec{H}+\vec{M})=\mu_o(1+\chi_m)\vec{H}$$
        $$\mu=\mu_o(1+\chi_m)$$

      \item Unlike dielectrics, $\chi_m$ could be either positive or negative

      \item Paramagnetism signifies $\chi_m>0$, more exactly $10^{-6}\leq\chi_m\leq10^{-1}$, which means $\vec{m}$ aligns with $\vec{B}$

      \item Diamagnetic materials signify that $-10^{-9}\leq\chi_m\leq-10^{-4}$

      \item Ferromagnetism signifies that the domains of magnetic dipoles align with an external magnetic field, which strengthens the field

        \begin{itemize}

          \item Hysteresis causes the magnetic field to ``lag behind'' in a ferromagnetic, even when the inducing magnet/magnetic field is gone

        \end{itemize}

    \end{itemize}

\end{itemize}

\end{document}

