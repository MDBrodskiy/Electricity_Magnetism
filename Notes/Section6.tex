%%%%%%%%%%%%%%%%%%%%%%%%%%%%%%%%%%%%%%%%%%%%%%%%%%%%%%%%%%%%%%%%%%%%%%%%%%%%%%%%%%%%%%%%%%%%%%%%%%%%%%%%%%%%%%%%%%%%%%%%%%%%%%%%%%%%%%%%%%%%%%%%%%%%%%%%%%%%%%%%%%%
% Written By Michael Brodskiy
% Class: Electricity & Magnetism
% Professor: D. Wood
%%%%%%%%%%%%%%%%%%%%%%%%%%%%%%%%%%%%%%%%%%%%%%%%%%%%%%%%%%%%%%%%%%%%%%%%%%%%%%%%%%%%%%%%%%%%%%%%%%%%%%%%%%%%%%%%%%%%%%%%%%%%%%%%%%%%%%%%%%%%%%%%%%%%%%%%%%%%%%%%%%%

\include{Includes.tex}

\title{Magnetic Fields in Matter}
\date{\today}
\author{Michael Brodskiy\\ \small Professor: D. Wood}

\begin{document}

\maketitle

\begin{itemize}

  \item Torque on a Dipole

    $$\vec{N}=\vec{m}\times\vec{B}$$

    \begin{itemize}

      \item This is akin to the torque on an electric dipole:

        $$\vec{N}=\vec{p}\times\vec{E}$$

      \item We can calculate the torque to be:

        $$\vec{N}=m\vec{B}\sin(\theta)$$

      \item The energy can be defined as:

        $$U=-\vec{m}\cdot\vec{B}$$

      \item The bound bulk and surface currents can be defined as:

        $$\vec{J}_b=\vec{\nabla}\times\vec{M}\quad\text{ and }\quad\vec{K}_b=\vec{M}\times\bold{\hat{n}}$$

      \item For a uniform magnetized sphere:

        $$\vec{J}_b=0\quad\text{ and }\quad\vec{K}_b=M\sin(\theta)\hat{\phi}$$

    \end{itemize}

\end{itemize}

\end{document}

