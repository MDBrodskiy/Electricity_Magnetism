  %%%%%%%%%%%%%%%%%%%%%%%%%%%%%%%%%%%%%%%%%%%%%%%%%%%%%%%%%%%%%%%%%%%%%%%%%%%%%%%%%%%%%%%%%%%%%%%%%%%%%%%%%%%%%%%%%%%%%%%%%%%%%%%%%%%%%%%%%%%%%%%%%%%%%%%%%%%%%%%%%%%
% Written By Michael Brodskiy
% Class: Electricity & Magnetism
% Professor: D. Wood
%%%%%%%%%%%%%%%%%%%%%%%%%%%%%%%%%%%%%%%%%%%%%%%%%%%%%%%%%%%%%%%%%%%%%%%%%%%%%%%%%%%%%%%%%%%%%%%%%%%%%%%%%%%%%%%%%%%%%%%%%%%%%%%%%%%%%%%%%%%%%%%%%%%%%%%%%%%%%%%%%%%

\documentclass[12pt]{article} 
\usepackage{alphalph}
\usepackage[utf8]{inputenc}
\usepackage[russian,english]{babel}
\usepackage{titling}
\usepackage{amsmath}
\usepackage{graphicx}
\usepackage{enumitem}
\usepackage{amssymb}
\usepackage[super]{nth}
\usepackage{everysel}
\usepackage{ragged2e}
\usepackage{geometry}
\usepackage{multicol}
\usepackage{fancyhdr}
\usepackage{cancel}
\usepackage{siunitx}
\usepackage{physics}
\usepackage{tikz}
\usepackage{mathdots}
\usepackage{yhmath}
\usepackage{cancel}
\usepackage{color}
\usepackage{array}
\usepackage{multirow}
\usepackage{gensymb}
\usepackage{tabularx}
\usepackage{extarrows}
\usepackage{booktabs}
\usepackage{lastpage}
\usetikzlibrary{fadings}
\usetikzlibrary{patterns}
\usetikzlibrary{shadows.blur}
\usetikzlibrary{shapes}

\geometry{top=1.0in,bottom=1.0in,left=1.0in,right=1.0in}
\newcommand{\subtitle}[1]{%
  \posttitle{%
    \par\end{center}
    \begin{center}\large#1\end{center}
    \vskip0.5em}%

}
\usepackage{hyperref}
\hypersetup{
colorlinks=true,
linkcolor=blue,
filecolor=magenta,      
urlcolor=blue,
citecolor=blue,
}


\title{Electromagnetic Waves}
\date{\today}
\author{Michael Brodskiy\\ \small Professor: D. Wood}

\begin{document}

\maketitle

\begin{itemize}

  \item Maxwell's Equations

    $$\vec{\nabla}\cdot\vec{E}=\frac{\rho}{\varepsilon_o}$$
    $$\vec{\nabla}\cdot\vec{B}=-2$$
    $$\vec{\nabla}\times\vec{E}=-\frac{\partial\vec{B}}{\partial t}$$
    $$\vec{\nabla}\times\vec{B}=\mu_o\vec{J}+\mu_o\varepsilon_o\frac{\partial\vec{E}}{\partial t}$$

    \begin{itemize}

      \item The second equation indicates no magnetic monopoles exist

      \item Where $J_d=\displaystyle\oint\vec{B}\cdot d\vec{l}=\mu_oI$

      \item Electric fields diverge with strength proportional to charge

      \item Magnetic fields do not diverge

      \item Curling electric field opposes change in magnetic field

        $$\vec{F}=q(\vec{E}+\vec{v}\time\vec{B})$$

      \item Describes the Lorentz force, which indicates how charges move in fields

      \item A key term, the displacement current ($\vec{J}_d$), was added by Maxwell

        $$\vec{J}_d=\varepsilon_o\frac{\partial \vec{E}}{\partial t}=\frac{\partial \vec{D}}{\partial t}$$
        $$I_d=\int\vec{J}_d\,d\vec{a}$$

      \item Within a capacitor, the displacement current may be defined by:

        $$I_d=\varepsilon_o\frac{d\vec{E}}{dt}A$$

      \item In matter, the equivalent to $\vec{J}_d$ is polarization current

        $$\vec{J}_p=\frac{\partial\vec{P}}{\partial t}$$
        $$\vec{\nabla}\cdot\vec{J}_p=-\frac{\partial\rho_b}{\partial t}$$

      \item Thus, we can rewrite some of the equations:

        $$\vec{\nabla}\cdot\vec{E}=\frac{\rho_f}{\varepsilon_o}+\frac{-1}{\varepsilon_o}(\vec{\nabla}\cdot\vec{P})\rightarrow \vec{\nabla}\cdot\vec{D}=\rho_f$$
        $$\vec{\nabla}\times\vec{B}=\mu_o\vec{J}_f+\mu_o\varepsilon_o\frac{\partial\vec{E}}{\partial t}+\mu_o\frac{\partial\vec{P}}{\partial t}+\mu_o(\vec{\nabla}\times\vec{m})\rightarrow \vec{\nabla}\times\vec{H}=\vec{J}_f+\frac{\partial\vec{D}}{\partial t}$$

    \end{itemize}

  \item Electromagnetic Waves

    $$\vec{\nabla}(\vec{\nabla}\times\vec{E})=-\frac{\partial}{\partial t}(\vec{\nabla}\times\vec{B})=-\mu_o\varepsilon_o\frac{\partial}{\partial t}\left( \frac{\partial \vec{E}}{\partial t} \right)$$
    $$\vec{\nabla}^2\vec{E}=\mu_o\varepsilon_o\frac{\partial^2\vec{E}}{\partial t^2}$$

    \begin{itemize}

      \item Applying our knowledge of the wave equation, we may write:

        $$v=\frac{1}{\sqrt{\mu_o\varepsilon_o}}=c$$

      \item We obtain the same result:

        $$\vec{\nabla}^2\vec{E}=c^2\frac{\partial^2\vec{E}}{\partial t^2}$$
        $$\vec{\nabla}^2\vec{B}=c^2\frac{\partial^2\vec{B}}{\partial t^2}$$

      \item Magnetic and electric fields propagate at the speed of light, perpendicular to each other, to create light

      \item The direction of the wave is $\vec{E}\times\vec{B}$

    \end{itemize}

\end{itemize}

\end{document}

