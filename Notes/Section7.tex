%%%%%%%%%%%%%%%%%%%%%%%%%%%%%%%%%%%%%%%%%%%%%%%%%%%%%%%%%%%%%%%%%%%%%%%%%%%%%%%%%%%%%%%%%%%%%%%%%%%%%%%%%%%%%%%%%%%%%%%%%%%%%%%%%%%%%%%%%%%%%%%%%%%%%%%%%%%%%%%%%%%
% Written By Michael Brodskiy
% Class: Electricity & Magnetism
% Professor: D. Wood
%%%%%%%%%%%%%%%%%%%%%%%%%%%%%%%%%%%%%%%%%%%%%%%%%%%%%%%%%%%%%%%%%%%%%%%%%%%%%%%%%%%%%%%%%%%%%%%%%%%%%%%%%%%%%%%%%%%%%%%%%%%%%%%%%%%%%%%%%%%%%%%%%%%%%%%%%%%%%%%%%%%

\documentclass[12pt]{article} 
\usepackage{alphalph}
\usepackage[utf8]{inputenc}
\usepackage[russian,english]{babel}
\usepackage{titling}
\usepackage{amsmath}
\usepackage{graphicx}
\usepackage{enumitem}
\usepackage{amssymb}
\usepackage[super]{nth}
\usepackage{everysel}
\usepackage{ragged2e}
\usepackage{geometry}
\usepackage{multicol}
\usepackage{fancyhdr}
\usepackage{cancel}
\usepackage{siunitx}
\usepackage{physics}
\usepackage{tikz}
\usepackage{mathdots}
\usepackage{yhmath}
\usepackage{cancel}
\usepackage{color}
\usepackage{array}
\usepackage{multirow}
\usepackage{gensymb}
\usepackage{tabularx}
\usepackage{extarrows}
\usepackage{booktabs}
\usepackage{lastpage}
\usetikzlibrary{fadings}
\usetikzlibrary{patterns}
\usetikzlibrary{shadows.blur}
\usetikzlibrary{shapes}

\geometry{top=1.0in,bottom=1.0in,left=1.0in,right=1.0in}
\newcommand{\subtitle}[1]{%
  \posttitle{%
    \par\end{center}
    \begin{center}\large#1\end{center}
    \vskip0.5em}%

}
\usepackage{hyperref}
\hypersetup{
colorlinks=true,
linkcolor=blue,
filecolor=magenta,      
urlcolor=blue,
citecolor=blue,
}


\title{Electrodynamics}
\date{\today}
\author{Michael Brodskiy\\ \small Professor: D. Wood}

\begin{document}

\maketitle

\begin{itemize}

  \item Current
    
    \begin{itemize}

      \item Ohm's ``Law''\footnote{Note: this is not a fundamental law}

        \begin{itemize}

          \item Holds when there is some current density such that $\vec{J}=\sigma\vec{E}$, with $\sigma$ as conductivity

          \item The unit of conductivity is $\left[ \frac{\si{\ampere}}{\si{\volt\meter}} \right]$

          \item The resistivity is the inverse of the conductivity, $\rho=\sigma^{-1}$, with units $\left[ \si{\ohm\meter} \right]$\footnote{Note: Ohms are equal to $\frac{\si{\volt}}{\si{\ampere}}$}

        \end{itemize}

      \item The average velocity of a particle accelerated over an interval due to an electric field is:

        $$v_{avg}=\sqrt{\frac{q\vec{E}d}{2m}}$$

      \item The current density can be defined as

        $$\vec{J}=nq\vec{v}$$

      \item An electron's drift velocity may be defined as:

        $$v_d=\frac{1}{2}\frac{q\vec{E}d}{mv}$$

        \begin{itemize}

          \item As long as $v_d<<v$

        \end{itemize}

      \item Given a wire of length $L$ and potential $V_o$, we can calculate:

        $$\vec{E}=\frac{V_o}{L}$$
        $$R=\frac{V}{I}=\frac{\vec{E}L}{\vec{J}A}=\frac{\rho L}{A}$$

    \end{itemize}

\end{itemize}

\end{document}

