%%%%%%%%%%%%%%%%%%%%%%%%%%%%%%%%%%%%%%%%%%%%%%%%%%%%%%%%%%%%%%%%%%%%%%%%%%%%%%%%%%%%%%%%%%%%%%%%%%%%%%%%%%%%%%%%%%%%%%%%%%%%%%%%%%%%%%%%%%%%%%%%%%%%%%%%%%%%%%%%%%%
% Written By Michael Brodskiy
% Class: Electricity & Magnetism
% Professor: D. Wood
%%%%%%%%%%%%%%%%%%%%%%%%%%%%%%%%%%%%%%%%%%%%%%%%%%%%%%%%%%%%%%%%%%%%%%%%%%%%%%%%%%%%%%%%%%%%%%%%%%%%%%%%%%%%%%%%%%%%%%%%%%%%%%%%%%%%%%%%%%%%%%%%%%%%%%%%%%%%%%%%%%%

\documentclass[12pt]{article} 
\usepackage{alphalph}
\usepackage[utf8]{inputenc}
\usepackage[russian,english]{babel}
\usepackage{titling}
\usepackage{amsmath}
\usepackage{graphicx}
\usepackage{enumitem}
\usepackage{amssymb}
\usepackage[super]{nth}
\usepackage{everysel}
\usepackage{ragged2e}
\usepackage{geometry}
\usepackage{multicol}
\usepackage{fancyhdr}
\usepackage{cancel}
\usepackage{siunitx}
\usepackage{physics}
\usepackage{tikz}
\usepackage{mathdots}
\usepackage{yhmath}
\usepackage{cancel}
\usepackage{color}
\usepackage{array}
\usepackage{multirow}
\usepackage{gensymb}
\usepackage{tabularx}
\usepackage{extarrows}
\usepackage{booktabs}
\usepackage{lastpage}
\usetikzlibrary{fadings}
\usetikzlibrary{patterns}
\usetikzlibrary{shadows.blur}
\usetikzlibrary{shapes}

\geometry{top=1.0in,bottom=1.0in,left=1.0in,right=1.0in}
\newcommand{\subtitle}[1]{%
  \posttitle{%
    \par\end{center}
    \begin{center}\large#1\end{center}
    \vskip0.5em}%

}
\usepackage{hyperref}
\hypersetup{
colorlinks=true,
linkcolor=blue,
filecolor=magenta,      
urlcolor=blue,
citecolor=blue,
}


\title{Electrodynamics}
\date{\today}
\author{Michael Brodskiy\\ \small Professor: D. Wood}

\begin{document}

\maketitle

\begin{itemize}

  \item Current
    
    \begin{itemize}

      \item Ohm's ``Law''\footnote{Note: this is not a fundamental law}

        \begin{itemize}

          \item Holds when there is some current density such that $\vec{J}=\sigma\vec{E}$, with $\sigma$ as conductivity

          \item The unit of conductivity is $\left[ \frac{\si{\ampere}}{\si{\volt\meter}} \right]$

          \item The resistivity is the inverse of the conductivity, $\rho=\sigma^{-1}$, with units $\left[ \si{\ohm\meter} \right]$\footnote{Note: Ohms are equal to $\frac{\si{\volt}}{\si{\ampere}}$}

        \end{itemize}

      \item The average velocity of a particle accelerated over an interval due to an electric field is:

        $$v_{avg}=\sqrt{\frac{q\vec{E}d}{2m}}$$

      \item The current density can be defined as

        $$\vec{J}=nq\vec{v}$$

      \item An electron's drift velocity may be defined as:

        $$v_d=\frac{1}{2}\frac{q\vec{E}d}{mv}$$

        \begin{itemize}

          \item As long as $v_d<<v$

        \end{itemize}

      \item Given a wire of length $L$ and potential $V_o$, we can calculate:

        $$\vec{E}=\frac{V_o}{L}$$
        $$R=\frac{V}{I}=\frac{\vec{E}L}{\vec{J}A}=\frac{\rho L}{A}$$

    \end{itemize}

  \item Circuits and Power

    \begin{itemize}

      \item We know:

        $$V=\frac{Q}{C}$$
        $$W=\frac{1}{2}QV=\frac{Q^2}{2C}$$

      \item By conservation of charge, we can write:

        $$P=\frac{dW}{dt}=\frac{1}{2C}\frac{d}{dt}(Q^2)=-IV$$
        $$P=\frac{V^2}{R}$$

      \item We can also derive:

        $$\frac{dQ}{Q}=-\frac{dt}{RC}$$
        $$Q=Q_oe^{-\frac{t}{RC}}$$

    \end{itemize}

  \item Electromotive Force (EMF)

    \begin{itemize}

      \item The EMF can be defined as:

        $$\varepsilon=\int\vec{f}\,d\vec{l}$$

      \item Where:

        $$\vec{f}=\frac{\vec{F}}{q}$$

      \item Magnetic flux can be defined as:

        $$\Phi=\int\vec{B}\cdot d\vec{a}$$

      \item The EMF can also be defined as:

        $$\varepsilon=-\frac{d\Phi}{dt}$$

      \item Lenz's Law: Induced effect opposes the change

      \item There are several ways flux may be changed:

        \begin{itemize}

          \item Loop is stationary, move $B$-field

          \item Loop stationary, change strength of $B$-field

          \item Change relative direction of loop and $\vec{B}$

        \end{itemize}

    \end{itemize}

  \item Faraday's Law:

    $$\oint\vec{E}\cdot d\vec{l}=-\frac{d}{dt}\int_S\vec{B}\cdot d\vec{a}$$

    \begin{itemize}

      \item According to Stokes' Theorem, we may write:

        $$\int_S(\vec{\nabla}\times\vec{E})\cdot d\vec{a}=-\int_S\frac{d}{dt}(\vec{B}\cdot d\vec{a})$$

      \item Which gives us one of Maxwell's equations:

        $$\vec{\nabla}\times\vec{E}=-\frac{d\vec{B}}{dt}$$

    \end{itemize}

  \item We can define flux as:

    $$\Phi_B(t)=BA\cos(\omega t)$$

    \begin{itemize}

      \item Which then gives us:

        $$\varepsilon=-\frac{d}{dt}(\Phi_B(t))=AB\omega\sin(\omega t)$$

      \item When the magnetic field is constant, but the area is changing, we can write:

        $$\varepsilon=-B\frac{dA}{dt}$$

      \item Thus, for a moving loop, we can write:

        $$\varepsilon=-Bwv$$

    \end{itemize}

  \item Mutual Inductance

    \begin{itemize}

      \item Since $\vec{B}$ is proportional to $I$ (via Biot-Savart), we can also say that $\Phi$ will be proportional to the current $I$. Thus, we may write:

        $$\Phi=MI$$

      \item Where $M$ is known as the mutual inductance

      \item Likewise, we can define:

        $$\varepsilon=-M\frac{dI}{dt}$$

      \item We can observe:

        \begin{enumerate}

          \item $\Phi$ is proportional to $I$

          \item $M$ depends only on the geometry

          \item $M_{1,2}=M_{2,1}=M$

        \end{enumerate}

    \end{itemize}

  \item Self Inductance

    $$\Phi=LI$$

    \begin{itemize}

      \item $L$ describes the self inductance

    \end{itemize}

\end{itemize}

\end{document}

